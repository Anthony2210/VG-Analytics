\documentclass[mstat,12pt]{unswthesis}

\usepackage{color}
\usepackage{fancyvrb}
\newcommand{\VerbBar}{|}
\newcommand{\VERB}{\Verb[commandchars=\\\{\}]}
\DefineVerbatimEnvironment{Highlighting}{Verbatim}{commandchars=\\\{\}}
% Add ',fontsize=\small' for more characters per line
\usepackage{framed}
\definecolor{shadecolor}{RGB}{248,248,248}
\newenvironment{Shaded}{\begin{snugshade}}{\end{snugshade}}
\newcommand{\AlertTok}[1]{\textcolor[rgb]{0.94,0.16,0.16}{#1}}
\newcommand{\AnnotationTok}[1]{\textcolor[rgb]{0.56,0.35,0.01}{\textbf{\textit{#1}}}}
\newcommand{\AttributeTok}[1]{\textcolor[rgb]{0.13,0.29,0.53}{#1}}
\newcommand{\BaseNTok}[1]{\textcolor[rgb]{0.00,0.00,0.81}{#1}}
\newcommand{\BuiltInTok}[1]{#1}
\newcommand{\CharTok}[1]{\textcolor[rgb]{0.31,0.60,0.02}{#1}}
\newcommand{\CommentTok}[1]{\textcolor[rgb]{0.56,0.35,0.01}{\textit{#1}}}
\newcommand{\CommentVarTok}[1]{\textcolor[rgb]{0.56,0.35,0.01}{\textbf{\textit{#1}}}}
\newcommand{\ConstantTok}[1]{\textcolor[rgb]{0.56,0.35,0.01}{#1}}
\newcommand{\ControlFlowTok}[1]{\textcolor[rgb]{0.13,0.29,0.53}{\textbf{#1}}}
\newcommand{\DataTypeTok}[1]{\textcolor[rgb]{0.13,0.29,0.53}{#1}}
\newcommand{\DecValTok}[1]{\textcolor[rgb]{0.00,0.00,0.81}{#1}}
\newcommand{\DocumentationTok}[1]{\textcolor[rgb]{0.56,0.35,0.01}{\textbf{\textit{#1}}}}
\newcommand{\ErrorTok}[1]{\textcolor[rgb]{0.64,0.00,0.00}{\textbf{#1}}}
\newcommand{\ExtensionTok}[1]{#1}
\newcommand{\FloatTok}[1]{\textcolor[rgb]{0.00,0.00,0.81}{#1}}
\newcommand{\FunctionTok}[1]{\textcolor[rgb]{0.13,0.29,0.53}{\textbf{#1}}}
\newcommand{\ImportTok}[1]{#1}
\newcommand{\InformationTok}[1]{\textcolor[rgb]{0.56,0.35,0.01}{\textbf{\textit{#1}}}}
\newcommand{\KeywordTok}[1]{\textcolor[rgb]{0.13,0.29,0.53}{\textbf{#1}}}
\newcommand{\NormalTok}[1]{#1}
\newcommand{\OperatorTok}[1]{\textcolor[rgb]{0.81,0.36,0.00}{\textbf{#1}}}
\newcommand{\OtherTok}[1]{\textcolor[rgb]{0.56,0.35,0.01}{#1}}
\newcommand{\PreprocessorTok}[1]{\textcolor[rgb]{0.56,0.35,0.01}{\textit{#1}}}
\newcommand{\RegionMarkerTok}[1]{#1}
\newcommand{\SpecialCharTok}[1]{\textcolor[rgb]{0.81,0.36,0.00}{\textbf{#1}}}
\newcommand{\SpecialStringTok}[1]{\textcolor[rgb]{0.31,0.60,0.02}{#1}}
\newcommand{\StringTok}[1]{\textcolor[rgb]{0.31,0.60,0.02}{#1}}
\newcommand{\VariableTok}[1]{\textcolor[rgb]{0.00,0.00,0.00}{#1}}
\newcommand{\VerbatimStringTok}[1]{\textcolor[rgb]{0.31,0.60,0.02}{#1}}
\newcommand{\WarningTok}[1]{\textcolor[rgb]{0.56,0.35,0.01}{\textbf{\textit{#1}}}}


%%%%%%%%%%%%%%%%%%%%%%%%%%%%%%%%%%%%%%%%%%%%%%%%%%%%%%%%%%%%%%%%%%
% 
% OK...Now we get to some actual input.  The first part sets up
% the title etc that will appear on the front page
%
%%%%%%%%%%%%%%%%%%%%%%%%%%%%%%%%%%%%%%%%%%%%%%%%%%%%%%%%%%%%%%%%%

\title{Projet réalisé par l'équipe \\
\strut \\
\textbf{Groupe 17 Vidéo Games Analytics}\\[0.5cm]Rapport de groupe en
Sciences des Données 2 + Bases de données}

\authornameonly{Anthony COMBES--AGUÉRA (22113542), Mohamed Chaouki
REKHIS (22212667), Raphaël BAYET (22206475), Yanick TINAUT (22212676) }

\author{\Authornameonly}

\copyrightfalse
\figurespagefalse
\tablespagefalse

%%%%%%%%%%%%%%%%%%%%%%%%%%%%%%%%%%%%%%%%%%%%%%%%%%%%%%%%%%%%%%%%%
%
%  And now the document begins
%  The \beforepreface and \afterpreface commands puts the
%  contents page etc in
%
%%%%%%%%%%%%%%%%%%%%%%%%%%%%%%%%%%%%%%%%%%%%%%%%%%%%%%%%%%%%%%%%%%


%%%%%%%%%%%%%%%%%%%%%%%%%%%%%%%%%%%%%%%%%%%%%%%%%%%%%%%%%%%%%%%%%%%%%%%
%
%  A small sample UNSW Coursework Masters thesis file.
%  Any questions to Ian Doust i.doust@unsw.edu.au and/or Gery Geenens ggeenens@unsw.edu.au
%
%%%%%%%%%%%%%%%%%%%%%%%%%%%%%%%%%%%%%%%%%%%%%%%%%%%%%%%%%%%%%%%%%%%%%%%
%
%  The first part pulls in a UNSW Thesis class file.  This one is
%  slightly nonstandard and has been set up to do a couple of
%  things automatically
%
 
%%%%%%%%%%%%%%%%%
%% Precisely one of the next four lines should be uncommented.
%% Choose the one which matches your degree, uncomment it, and comment out the other two!
%\documentclass[mfin,12pt]{unswthesis}    %%  For Master of Financial Mathematics 
%\documentclass[mmath,12pt]{unswthesis}   %%  For Master of Mathematics
%\documentclass[mstat,12pt]{unswthesis}  %%  For Master of Statistics
%%%%%%%%%%%%%%%%%



\linespread{1}
\usepackage{amsfonts}
\usepackage{amssymb}
\usepackage{amsthm}
\usepackage{latexsym,amsmath}
\usepackage{graphicx}
\usepackage{afterpage}
\usepackage[colorlinks]{hyperref}
\usepackage{longtable}
\usepackage{booktabs}
\usepackage{array}
\usepackage{geometry}
\geometry{left=2.5cm,right=2.5cm,top=2.5cm,bottom=2.5cm}
 \hypersetup{
     colorlinks=true,
     linkcolor=blue,
     filecolor=blue,
     citecolor= black,      
     urlcolor=cyan,
     }
\usepackage{textcomp}
\usepackage{longtable}
\usepackage{booktabs}
\usepackage{float}
\let\origfigure\figure
\let\endorigfigure\endfigure
\renewenvironment{figure}[1][2] {
    \expandafter\origfigure\expandafter[H]
} {
    \endorigfigure
}
\usepackage[T1]{fontenc}
\usepackage{ragged2e}
\def\tightlist{}

%%%%%%%%%%%%%%%%%%%%%%%%%%%%%%%%%%%%%%%%%%%%%%%%%%%%%%%%%%%%%%%%%
%
%  The following are some simple LaTeX macros to give some
%  commonly used letters in funny fonts. You may need more or less of
%  these
%
\newcommand{\R}{\mathbb{R}}
\newcommand{\Q}{\mathbb{Q}}
\newcommand{\C}{\mathbb{C}}
\newcommand{\N}{\mathbb{N}}
\newcommand{\F}{\mathbb{F}}
\newcommand{\PP}{\mathbb{P}}
\newcommand{\T}{\mathbb{T}}
\newcommand{\Z}{\mathbb{Z}}
\newcommand{\B}{\mathfrak{B}}
\newcommand{\BB}{\mathcal{B}}
\newcommand{\M}{\mathfrak{M}}
\newcommand{\X}{\mathfrak{X}}
\newcommand{\Y}{\mathfrak{Y}}
\newcommand{\CC}{\mathcal{C}}
\newcommand{\E}{\mathbb{E}}
\newcommand{\cP}{\mathcal{P}}
\newcommand{\cS}{\mathcal{S}}
\newcommand{\A}{\mathcal{A}}
\newcommand{\ZZ}{\mathcal{Z}}
%%%%%%%%%%%%%%%%%%%%%%%%%%%%%%%%%%%%%%%%%%%%%%%%%%%%%%%%%%%%%%%%%%%%%
%
% The following are much more esoteric commands that I have left in
% so that this file still processes. Use or delete as you see fit
%
\newcommand{\bv}[1]{\mbox{BV($#1$)}}
\newcommand{\comb}[2]{\left(\!\!\!\begin{array}{c}#1\\#2\end{array}\!\!\!\right)
}
\newcommand{\Lat}{{\rm Lat}}
\newcommand{\var}{\mathop{\rm var}}
\newcommand{\Pt}{{\mathcal P}}
\def\tr(#1){{\rm trace}(#1)}
\def\Exp(#1){{\mathbb E}(#1)}
\def\Exps(#1){{\mathbb E}\sparen(#1)}
\newcommand{\floor}[1]{\left\lfloor #1 \right\rfloor}
\newcommand{\ceil}[1]{\left\lceil #1 \right\rceil}
\newcommand{\hatt}[1]{\widehat #1}
\newcommand{\modeq}[3]{#1 \equiv #2 \,(\text{mod}\, #3)}
\newcommand{\rmod}{\,\mathrm{mod}\,}
\newcommand{\p}{\hphantom{+}}
\newcommand{\vect}[1]{\mbox{\boldmath $ #1 $}}
\newcommand{\reff}[2]{\ref{#1}.\ref{#2}}
\newcommand{\psum}[2]{\sum_{#1}^{#2}\!\!\!'\,\,}
\newcommand{\bin}[2]{\left( \begin{array}{@{}c@{}}
				#1 \\ #2
			\end{array}\right)	}
%
%  Macros - some of these are in plain TeX (gasp!)
%
\newcommand{\be}{($\beta$)}
\newcommand{\eqp}{\mathrel{{=}_p}}
\newcommand{\ltp}{\mathrel{{\prec}_p}}
\newcommand{\lep}{\mathrel{{\preceq}_p}}
\def\brack#1{\left \{ #1 \right \}}
\def\bul{$\bullet$\ }
\def\cl{{\rm cl}}
\let\del=\partial
\def\enditem{\par\smallskip\noindent}
\def\implies{\Rightarrow}
\def\inpr#1,#2{\t \hbox{\langle #1 , #2 \rangle} \t}
\def\ip<#1,#2>{\langle #1,#2 \rangle}
\def\lp{\ell^p}
\def\maxb#1{\max \brack{#1}}
\def\minb#1{\min \brack{#1}}
\def\mod#1{\left \vert #1 \right \vert}
\def\norm#1{\left \Vert #1 \right \Vert}
\def\paren(#1){\left( #1 \right)}
\def\qed{\hfill \hbox{$\Box$} \smallskip}
\def\sbrack#1{\Bigl \{ #1 \Bigr \} }
\def\ssbrack#1{ \{ #1 \} }
\def\smod#1{\Bigl \vert #1 \Bigr \vert}
\def\smmod#1{\bigl \vert #1 \bigr \vert}
\def\ssmod#1{\vert #1 \vert}
\def\sspmod#1{\vert\, #1 \, \vert}
\def\snorm#1{\Bigl \Vert #1 \Bigr \Vert}
\def\ssnorm#1{\Vert #1 \Vert}
\def\sparen(#1){\Bigl ( #1 \Bigr )}

\newcommand\blankpage{%
    \null
    \thispagestyle{empty}%
    \addtocounter{page}{-1}%
    \newpage}

%%%%%%%%%%%%%%%%%%%%%%%%%%%%%%%
%
% These environments allow you to get nice numbered headings
%  for your Theorems, Definitions etc.  
%
%  Environments
%
%%%%%%%%%%%%%%%%%%%%%%%%%%%%%%%

\newtheorem{theorem}{Theorem}[section]
\newtheorem{lemma}[theorem]{Lemma}
\newtheorem{proposition}[theorem]{Proposition}
\newtheorem{corollary}[theorem]{Corollary}
\newtheorem{conjecture}[theorem]{Conjecture}
\newtheorem{definition}[theorem]{Definition}
\newtheorem{example}[theorem]{Example}
\newtheorem{remark}[theorem]{Remark}
\newtheorem{question}[theorem]{Question}
\newtheorem{notation}[theorem]{Notation}
\numberwithin{equation}{section}

%%%%%%%%%%%%%%%%%%%%%%%%%%%%%%%%%%%%%%%%%%%%%%%%%%%%%%%%%%%%%%%%%%
%
%  If you've got some funny special words that LaTeX might not
% hyphenate properly, you can give it a helping hand:
%

\hyphenation{Mar-cin-kie-wicz Rade-macher}


\newlength{\cslhangindent}
\setlength{\cslhangindent}{1.5em}
\newlength{\csllabelwidth}
\setlength{\csllabelwidth}{3em}
\newenvironment{CSLReferences}[2] % #1 hanging-ident, #2 entry spacing
 {% don't indent paragraphs
  \setlength{\parindent}{0pt}
  % turn on hanging indent if param 1 is 1
  \ifodd #1 \everypar{\setlength{\hangindent}{\cslhangindent}}\ignorespaces\fi
  % set entry spacing
  \ifnum #2 > 0
  \setlength{\parskip}{#2\baselineskip}
  \fi
 }%
 {}
\usepackage{calc} % for \widthof, \maxof
\newcommand{\CSLBlock}[1]{#1\hfill\break}
\newcommand{\CSLLeftMargin}[1]{\parbox[t]{\maxof{\widthof{#1}}{\csllabelwidth}}{#1}}
\newcommand{\CSLRightInline}[1]{\parbox[t]{\linewidth}{#1}}
\newcommand{\CSLIndent}[1]{\hspace{\cslhangindent}#1}






\renewcommand{\contentsname}{Table des matières}

\renewcommand{\chaptername}{Chapitre}



\begin{document}

\beforepreface

%\afterpage{\blankpage}

% plagiarism

\prefacesection{Déclaration de non plagiat}

\vskip 2pc \noindent Nous déclarons que ce rapport est le fruit de notre seul travail, à part lorsque cela est indiqué  explicitement. 

\vskip 2pc  \noindent Nous acceptons que la personne évaluant ce rapport puisse, pour les besoins de cette évaluation:
\begin{itemize}
\item la reproduire et en fournir une copie à un autre membre de l'université; et/ou,
\item en communiquer une copie à un service en ligne de détection de plagiat (qui pourra en retenir une copie pour les besoins d'évaluation future).
\end{itemize}

\vskip 2pc \noindent Nous certifions que nous avons lu et compris les règles ci-dessus.\vspace{24pt}

\vskip 2pc \noindent En signant cette déclaration, nous acceptons ce qui précède.
\vskip 2pc \noindent
Signature: \rule{7cm}{0.25pt} \hfill Date: \rule{4cm}{0.25pt} \\[1cm]
Signature: \rule{7cm}{0.25pt} \hfill Date: \rule{4cm}{0.25pt} \\[1cm]
Signature: \rule{7cm}{0.25pt} \hfill Date: \rule{4cm}{0.25pt} \\[1cm]
Signature: \rule{7cm}{0.25pt} \hfill Date: \rule{4cm}{0.25pt} \\[1cm]
\vskip 1pc

%\afterpage{\blankpage}

% Acknowledgements are optional


\prefacesection{Remerciements}

{\bigskip}Nos plus sincères remerciements vont à notre encadrant
pédagogique pour les conseils avisés sur notre travail.\\[1cm] Nous
remercions aussi Ayoub AKKOUH pour son aide psycologique durant cette
période compliquée pour nous.\\[1cm] 

{\bigskip\bigskip\bigskip\noindent} 30/04/2024.

%\afterpage{\blankpage}

% Abstract

\prefacesection{Résumé}

Dans notre rapport, nous avons examiné l'influence du prix initial et
des critiques sur le nombre de joueurs actifs pour des jeux sur Steam.
Notre analyse portant sur 950 jeux depuis leur lancement jusqu'en
juillet 2022 a révélé que ceux notés au-dessus de 75\% par la presse
voient leur base de joueurs augmenter en moyenne de 20\%, indépendamment
du prix. Pour les jeux à moins de 20 euros, un score élevé peut
quadrupler le nombre de joueurs actifs durant le premier mois après la
sortie.

Ces observations démontrent l'importance des critiques, surtout pour les
jeux à bas prix. Les stratégies de tarification et de marketing doivent
donc être ajustées en prévision de l'accueil des évaluateurs. Pour
approfondir, des analyses futures pourraient explorer l'effet des genres
de jeux ou des mises à jour post-lancement sur l'engagement des joueurs.

Nous préconisons deux perspectives : à court terme, améliorer
l'exactitude des prévisions du nombre de joueurs en intégrant plus de
variables; à long terme, étudier l'impact des tendances de l'industrie
du jeu, telles que les jeux en tant que service, les abonnements, les
microtransactions au sein d'un jeu initialement gratuit.

Nos difficultés résidaient principalement dans le choix et la
manipulation des bases de données. Les changements fréquents de dataset
et les ajustements de notre question de recherche ont été motivés par la
quête de données adaptées à nos besoins. Le partage des données
post-traitement a également présenté des défis, résolus en optant pour
GitHub, malgré son interface peu intuitive au départ. Le nettoyage des
données n'a pas posé de problème significatif, mais la standardisation
des noms de jeux et la conversion des dates en valeurs numériques ont
demandé un effort considérable. Enfin, l'apprentissage et l'application
des concepts théoriques, comme la rédaction en R Markdown, ont exigé une
grande dynamique d'apprentissage de notre part.

Ce projet a été une opportunité d'approfondir nos compétences pratiques
en science des données et nous a permis de réaliser un rendu qui nous
tient à cœur.

%\afterpage{\blankpage}


\afterpreface





%%%%%%%%%%%%%%%%%%%%%%%%%%%%%%%%%%%%%%%%%%%%%%%%%%%%%%%%%%%%%%%%%%
%
% Now we can start on the first chapter
% Within chapters we have sections, subsections and so forth
%
%%%%%%%%%%%%%%%%%%%%%%%%%%%%%%%%%%%%%%%%%%%%%%%%%%%%%%%%%%%%%%%%%%



%%%%%%%%%%%%%%%%%%%%%%%%%%%%%%%%%%%%%

%\afterpage{\blankpage}


\hypertarget{introduction}{%
\chapter{Introduction}\label{introduction}}

\hypertarget{pruxe9sentation}{%
\section{Présentation}\label{pruxe9sentation}}

Dans une ère qui se dirige de plus en plus vers le numérique, le secteur
des jeux vidéo est plus que jamais au cœur des transformations
économiques et culturelles. Alors que le monde du réel et celui du
virtuel tendent à se confondre, l'univers du jeu vidéo, permet
d'accroître de manière exponentielle, l'expérience récréative que
peuvent ressentir les joueurs.

De ses modestes débuts, dans les années 1950, à son statut actuel de
géant du divertissement, l'industrie du jeu vidéo a su faire preuve
d'une innovation constante dans la manière dont l'humanité interagit
avec le numérique, afin de devenir l'un des secteurs le plus lucratif de
l'économie mondiale.

Enfin, pour les développeurs et les distributeurs, engendrer des revenus
substantiels et maintenir une communauté de joueurs fidèles sont
devenues des objectifs centraux lorsqu'il s'agit de produire un jeu.
Face à cette réalité, notre projet ce concentre sur l'impact que peut
avoir le prix et les appréciations de la critique d'un jeu vidéo sur la
popularité de celui-ci.

Notre question de recherche est la suivante :

\bigskip

\centering

\textbf{Comment le nombre de joueurs actifs pour un jeu évolue-t-il par
rapport à son prix initial et à la note que la presse lui attribue, au
sein de la communauté de Steam ?}

\bigskip
\justifying

Ainsi, il s'agirait de savoir si un jeu vidéo ayant un prix initial
faible et une note élevée, permettrait de mieux maintenir une base de
joueur actifs dans le temps, que les autres jeux.

L'importance de cette question réside dans son implication pour
l'industrie du jeu vidéo, un secteur qui continue de croître et
d'évoluer à une vitesse vertigineuse. Comprendre ce que les critiques
des différents médias pourraient apporter sur les ventes et le choix des
joueurs. Ceci permettrait d'offrir des insights précieux pour les
développeurs et les marketeurs dans leur stratégie de création, de vente
et de promotion de leurs jeux. En outre, cette recherche peut éclairer
les consommateurs sur les dynamiques qui influencent la popularité et le
succès commercial des jeux vidéo. \newpage

\hypertarget{responsabilituxe9s-et-composition-de-luxe9quipe}{%
\section{Responsabilités et composition de
l'équipe}\label{responsabilituxe9s-et-composition-de-luxe9quipe}}

Nous nous sommes réparti les rôles pour couvrir tous les aspects
nécessaires à l'accomplissement de ce projet, de la collecte des données
à la rédaction du rapport.

\begin{itemize}
\tightlist
\item
  \textbf{Yanick Tinaut} a découvert le premier jeu de données intitulé
  `Steam Games' sur le site de Hugging Face. Il a également rédigé la
  première version (la V1) de la partie consacrée au cours de base de
  données sur le site Overleaf. Il a rédigé l'ensemble des requêtes SQL.
  Avec \textbf{Anthony Combes-Aguera}, ils ont pris en charge la
  rédaction des chapitres 2 et 3, qui traitent des jeux de données,
  ainsi que des sections sur l'interprétation des résultats, la
  conclusion et les perspectives que pourrait offrir cette étude.
  \smallskip
\item
  \textbf{Anthony Combes-Aguera} s'est quant à lui, occupé du
  prétraitement et du nettoyage des jeux de données à l'aide de la
  plateforme R Studio. Il a également importé les jeux de données sur la
  plateforme phpMyAdmin et implémenté les requêtes SQL directement dans
  le fichier Rmarkdown. Avec \textbf{Yanick Tinaut}, ils ont écrit les
  commentaires et les interprétations de ces résultats ainsi que la
  correction totale de l'ensemble des graphiques. Anthony est également
  le créateur de notre diaporama et de son contenu. \smallskip
\item
  \textbf{Raphaël Bayet} a découvert le second jeu de données `Steam
  Chart' sur le site Kaggle. Avec \textbf{Mohamed Rekhis Chaouki}, ils
  ont réalisé l'analyse statistique descriptive. Ils ont créé les
  graphiques et rédigé les interprétations correspondantes, puis les ont
  inclus dans le fichier Rmarkdown final. \smallskip
\item
  \textbf{Mohamed Rekhis Chaouki} s'est vu chargé de produire le test
  ANOVA.
\end{itemize}

\bigskip

Malgré le fait que nous avions réparti les différentes tâches entre les
membres, nous nous sommes entraidés sur les différents aspects du
projet, ce qui a non seulement rendu notre travail plus agréable, mais
également rendu celui-ci, beaucoup plus solide et cohérent.

\hypertarget{base-de-donnuxe9es}{%
\chapter{Base de données}\label{base-de-donnuxe9es}}

\hypertarget{provenance-des-donnuxe9es}{%
\section{Provenance des données}\label{provenance-des-donnuxe9es}}

Pour notre étude sur l'impact des évaluations et du prix initial des
jeux vidéo sur la longévité du nombre de joueurs actifs sur ceux-ci,
nous avons utilisé deux principaux jeux de données, disponibles en ligne
sur le site \textbf{Kaggle}\footnote{Kaggle, disponible sur
  \url{https://www.kaggle.com/datasets}.} et \textbf{Hugging
Face}\footnote{Hugging Face, disponible sur
  \url{https://huggingface.co}.}.

\begin{enumerate}
\def\labelenumi{\arabic{enumi}.}
\item
  \textbf{Steam Games Dataset}\footnote{Steam Games Dataset sur Hugging
    Face, disponible sur
    \url{https://huggingface.co/datasets/FronkonGames/steam-games-dataset}.}:
  Fournit une vue d'ensemble complète des jeux disponibles sur la
  plateforme Steam, en se concentrant sur des aspects tels que le prix
  de lancement, les genres, les notes des utilisateurs et de la presse,
  la moyenne de temps de jeu, et d'autres métadonnées essentielles. Ces
  informations permettent d'explorer les facteurs contribuant au succès
  initial et à la perception générale d'un jeu dans l'écosystème de
  Steam. En particulier, le prix de lancement peut être un indicateur
  crucial de la stratégie de positionnement d'un jeu sur le marché,
  tandis que les notes de Metacritic reflètent la réception
  communautaire et la satisfaction à l'égard de l'expérience de jeu.
\item
  \textbf{Player Counts on Steam}\footnote{Player Counts on Steam sur
    Kaggle, disponible sur
    \url{https://www.kaggle.com/datasets/josephvm/player-counts-on-steam}.}:
  Offre des données sur le nombre moyen de joueurs actifs par mois pour
  différents jeux sur Steam, permettant d'évaluer la fidélité ou la
  longévité du nombre de joueurs, dans le temps, pour un jeu donné. Ces
  valeurs sont des indicateurs clés de la réussite à long terme d'un
  jeu. En croisant ces données avec les notes et le prix des jeux vidéo,
  il est peut-être possible d'analyser comment ces deux variables
  affectent la capacité d'un jeu à maintenir une base de joueurs active
  dans le temps.
\end{enumerate}

\medskip

Ces bases de données ont été choisies pour leur complémentarité,
permettant une analyse approfondie de l'impact des évaluations et de son
prix d'achat sur la popularité à long terme des jeux vidéo. Nous avons
filtré les données pour nous concentrer sur la période de 2012 à 2020,
en ne sélectionnant que les jeux disposant des données de nombre de
joueurs actifs uniquement après leur date de sortie. La population
ciblée inclut donc les jeux vidéo lancés durant cette période, avec une
attention particulière portée aux titres ayant des données qui sont
cohérentes aux deux dataset. Pour réaliser cela, nous avons prétraité
les données des deux bases de données initiales avec le logiciel R
Studio.

\newpage

\hypertarget{descriptif-des-tables}{%
\section{Descriptif des tables}\label{descriptif-des-tables}}

\begin{enumerate}

\item\textbf{Steam Games Dataset}:
Ce jeu de données se présente sous la forme d'un fichier CSV de 244 Mo, comprenant 85 104 entrées, chacune représentant un jeu unique avec des informations détaillées réparties sur 39 colonnes. En combinant ces données avec des informations sur le nombre de joueurs actifs, il pourrait être possible d'analyser des tendances comme l'impact du prix et de la satisfaction des utilisateurs sur la longévité et la popularité d'un jeu au sein de la communauté Steam.
\smallskip
\item\textbf{Player Counts on Steam}: Les données sont présentées dans un fichier CSV de 3.49 Mo comportant 999 lignes qui représentent des jeux uniques et 7 colonnes.
\end{enumerate}

\hypertarget{description-du-nettoyage-des-donnuxe9es}{%
\section{Description du nettoyage des
données}\label{description-du-nettoyage-des-donnuxe9es}}

Le nettoyage des données a été fait sur RStudio\footnote{Le code est
  disponible en annexe.}.

\hypertarget{bibliothuxe8ques-utilisuxe9es}{%
\subsection{Bibliothèques
utilisées}\label{bibliothuxe8ques-utilisuxe9es}}

\begin{itemize}
\tightlist
\item
  \texttt{dplyr} : utilisée pour la manipulation des data frames.
\item
  \texttt{stringr} : permet le traitement efficace des chaînes de
  caractères.
\item
  \texttt{lubridate} : simplifie la gestion des dates.
\end{itemize}

\hypertarget{chargement-des-donnuxe9es}{%
\subsection{Chargement des données}\label{chargement-des-donnuxe9es}}

Les données sont chargées depuis des fichiers CSV locaux. Les principaux
ensembles de données sont :

\begin{enumerate}
\item \texttt{games.csv} : contient les détails des jeux.
\item \texttt{steam\_charts.csv} : inclut les informations sur les comptages de joueurs.
\end{enumerate}

\hypertarget{renommage-et-pruxe9paration-des-colonnes}{%
\subsection{Renommage et préparation des
colonnes}\label{renommage-et-pruxe9paration-des-colonnes}}

Les colonnes de \texttt{steam\_charts} ont été renommées pour clarifier
leur usage (\texttt{App.ID} en \texttt{AppID} et \texttt{Game} en
\texttt{Name}). Les transformations incluent la conversion des
pourcentages en valeurs numériques et le remplacement des tirets par des
zéros pour corriger les erreurs de type.

\hypertarget{filtrage-des-jeux}{%
\subsection{Filtrage des jeux}\label{filtrage-des-jeux}}

Nous avons exclu les jeux en chinois simplifié pour nous concentrer sur
ceux ayant des titres en alphabet latin, ce qui était essentiel pour
garantir l'intégrité des données dans \texttt{steam\_charts}. La date de
sortie a été uniformisée pour faciliter les comparaisons.

\hypertarget{mise-en-commun-des-jeux}{%
\subsection{Mise en commun des jeux}\label{mise-en-commun-des-jeux}}

Un nettoyage supplémentaire a permis d'aligner les jeux dans nos
fichiers CSV en utilisant l'identifiant \texttt{AppID}.

\hypertarget{suxe9lection-des-colonnes}{%
\subsection{Sélection des colonnes}\label{suxe9lection-des-colonnes}}

Nous avons réalisé une sélection rigoureuse des colonnes, conservant
uniquement celles pertinentes pour l'analyse ultérieure. En tout, 30
variables ont été supprimées du `Steam Games Dataset'. Pour `Player
Counts on Steam', seule la variable \texttt{Game} a été supprimée, car
\texttt{AppID} sert de clé primaire plus efficace.

\hypertarget{organisation-finale-et-sauvegarde}{%
\subsection{Organisation finale et
sauvegarde}\label{organisation-finale-et-sauvegarde}}

Les données finales ont été organisées par \texttt{AppID} et
sauvegardées au format CSV pour une utilisation future dans l'analyse.
\newpage

\hypertarget{tableaux-synthuxe8se}{%
\subsection{Tableaux synthèse}\label{tableaux-synthuxe8se}}

\begin{longtable}[t]{l>{\raggedright\arraybackslash}p{1cm}>{\raggedright\arraybackslash}p{4cm}>{\raggedright\arraybackslash}p{4cm}}
\caption{\label{tab:recuperation donnée}Player Counts on Steam (950 x 9)}\\
\toprule
Nom.colonne & Type & Signification & Caractéristique\\
\midrule
AppID & Integer & Identifiant du jeu sur Steam & Clé primaire, unique\\
Name & String & Nom du jeu & Pas de valeurs manquantes\\
ReleaseDate & Date & Date de sortie du jeu & Pas de valeurs manquantes\\
Price & Float & Prix initial du jeu & Pas de valeurs manquantes\\
MetacriticScore & Integer & Evaluation moyenne des critiques & Peut être 0\\
\addlinespace
Positive & Integer & Nombre de votes positifs & Peut être 0\\
Negative & Integer & Nombre de votes négatifs & Peut être 0\\
AveragePlaytimeForever & Integer & Temps de jeu moyen & Temps en minutes\\
Developers & String & Développeurs du jeu & -\\
\bottomrule
\end{longtable}

\begin{longtable}[t]{>{\raggedright\arraybackslash}p{2cm}>{\raggedright\arraybackslash}p{1cm}>{\raggedright\arraybackslash}p{8cm}>{\raggedright\arraybackslash}p{3cm}}
\caption{\label{tab:steam-charts-table}Steam Charts on Steam (52287 x 5)}\\
\toprule
Nom.colonne & Type & Signification & Caractéristique\\
\midrule
AppID & Integer & Identifiant unique du jeu sur Steam & Clé primaire, unique\\
Month & String & Mois où les statistiques sont prises & Format de date\\
AvgPlayers & Float & Nombre moyen de joueurs actifs & Peut être 0\\
Gain & Float & Gain du nombre de joueurs par rapport
                    au mois précédent & -\\
PercentGain & Float & Pourcentage du gain du nombre de joueurs
                    par rapport au mois précédent & -\\
\addlinespace
PeakPlayers & Integer & Nombre maximal de joueurs actifs & Peut être 0\\
\bottomrule
\end{longtable}

\newpage

\hypertarget{les-moduxe8les-conceptuels}{%
\section{Les modèles conceptuels}\label{les-moduxe8les-conceptuels}}

\hypertarget{moduxe8le-mcd}{%
\subsection{Modèle MCD}\label{moduxe8le-mcd}}

Le modèle conceptuel de données a été élaboré pour visualiser les
relations entre nos différentes tables et assurer une structure logique
de notre base de données. Le MCD, conçu à l'aide de l'outil en ligne
Mocodo (\url{https://www.mocodo.net/}), montre comment les jeux vidéo,
leurs prix, leurs évaluations, et données de joueurs actifs sont
interconnectés, facilitant ainsi les analyses croisées nécessaires pour
répondre à notre question de recherche.

\begin{itemize}
\item
  MCD réalisé avec le logiciel Mocodo :

  \begin{figure}
  \hypertarget{uml}{%
  \centering
  \includegraphics[width=8cm,height=4cm]{Image/MCD.png}
  \caption{Relation de Jouabilité entre `JEU' et `STATISTIQUE' via
  l'association `JOUER'.}\label{uml}
  }
  \end{figure}
\end{itemize}

\hypertarget{moduxe8le-mod}{%
\subsection{Modèle MOD}\label{moduxe8le-mod}}

À partir de notre modèle conceptuel de données qui présente une relation
déséquilibrée entre nos deux entités, nous avons créé un modèle
relationnel avec le designer de phpMyAdmin
(\url{http://localhost/phpMyAdmin/}), qui matérialise les concepts et
les associations en une structure adaptée. Ce modèle est composé de deux
tables principales et d'une associations. Les informations sur les jeux
sont reliées aux statistiques de jeu correspondantes, avec l'identifiant
de l'application (AppID) servant de clé étrangère pour relier les deux
entités. La clé primaire de la seconde table est donc une clé composite,
composée des variables ``AppID'' et ``Month''.

\begin{itemize}
\item
  MOD réalisé avec le designer de phpMyAdmin :

  \begin{figure}
  \hypertarget{uml}{%
  \centering
  \includegraphics[width=14cm,height=6cm]{Image/MOD.png}
  \caption{Modèle de Données des Jeux et Statistiques.}\label{uml}
  }
  \end{figure}
\end{itemize}

\newpage

\hypertarget{requuxeates-ruxe9alisuxe9es}{%
\section{Requêtes réalisées}\label{requuxeates-ruxe9alisuxe9es}}

Dans un premier temps, nous avons exécuté des requêtes simples afin
d'explorer différents angles de recherche. Ces démarches préliminaires
nous ont permis de répondre à notre question de recherche grâce à des
requêtes ultérieures, plus complexes mais également plus pertinentes.
Les requêtes que nous considérons comme optimales sont les suivantes :

\hypertarget{requuxeate-1-les-moyennes}{%
\subsection{Requête 1 : Les moyennes}\label{requuxeate-1-les-moyennes}}

\begingroup\fontsize{10}{12}\selectfont

\begin{longtable}[t]{>{\raggedleft\arraybackslash}p{3cm}r}
\caption{\label{tab:results-table moyenneinvisible}Requête 1, Les moyennes}\\
\toprule
Moyenne des prix & Moyenne des notes Metacritic\\
\midrule
20.59 & 35.19\\
\bottomrule
\end{longtable}
\endgroup{}

Cette requête sert à avoir la moyenne générale du prix et du «
MetacriticScore » pour tous les jeux de notre base de données. Nous
pourrions nous basé sur ces informartions pour créer des catégories afin
de réaliser des statistiques descriptives. \bigskip

\hypertarget{requuxeate-2-les-catuxe9gories-des-prix}{%
\subsection{Requête 2 : Les catégories des
prix}\label{requuxeate-2-les-catuxe9gories-des-prix}}

\begingroup\fontsize{10}{12}\selectfont

\begin{longtable}[t]{>{\raggedright\arraybackslash}p{4cm}r}
\caption{\label{tab:results-table catégorie prix invisible}Requête 2, Les catégories des prix}\\
\toprule
Catégorie de Prix & Moyenne des joueurs actifs\\
\midrule
Gratuit & 16146.02\\
Inférieur à 20 & 2512.70\\
Supérieur à 20 & 3625.50\\
\bottomrule
\end{longtable}
\endgroup{}

Cette requête SQL est conçue pour segmenter les jeux vidéo en trois
catégories de prix : gratuits, prix inférieurs à 20 euros et prix
supérieurs à 20 euros. Pour chaque catégorie, la requête calcule la
moyenne du nombre de joueurs actif (AvgPlayers).

En analysant ces résultats, on peut conclure que les jeux gratuits ont
significativement plus de joueurs actifs en moyenne que les jeux
payants, ce qui pourrait indiquer que l'absence de barrière financière à
l'entrée est un facteur majeur d'attraction pour une plus grande base de
joueurs. Cela est cohérent avec certaines tendances de l'industrie où
les modèles free-to-play peuvent attirer de larges publics. Les jeux
avec un prix supérieur à 20 euros ont également une base de joueurs
actifs plus importante en moyenne que les jeux à moins de 20 euros, ce
qui pourrait suggérer que les jeux à prix plus élevé possèdent soit une
qualité ou une réputation qui attire les joueurs, soit qu'ils
s'adressent à des niches de marché spécifiques prêtes à payer un premium
pour des expériences meilleures. Cependant, ces résultats doivent être
interprétés avec prudence, car d'autres facteurs (tels que la qualité du
jeu, le marketing, la popularité de la franchise, le genre du jeu, etc.)
peuvent également influencer le nombre de joueurs actifs et ne sont pas
pris en compte dans notre analyse.

\newpage

\hypertarget{requuxeate-3-les-developpeurs}{%
\subsection{Requête 3 : Les
developpeurs}\label{requuxeate-3-les-developpeurs}}

\begingroup\fontsize{10}{12}\selectfont

\begin{longtable}[t]{>{\raggedright\arraybackslash}p{5cm}rr}
\caption{\label{tab:results-table developpeurs invisible}Requête 3, Les developpeurs}\\
\toprule
Développeur & Moyenne des notes Metacritic & Moyenne des gains de joueurs\\
\midrule
ZA/UM & 97 & 39.95\\
Beat Games & 93 & 18.07\\
Relic Entertainment & 93 & 4.23\\
Wube Software LTD. & 90 & 115.09\\
Firaxis Games & 90 & 5.70\\
\addlinespace
ConcernedApe & 89 & 403.67\\
Mega Crit Games & 89 & 136.00\\
Motion Twin & 89 & 36.72\\
SkyBox Labs,Big Huge Games & 89 & 1.19\\
Cellar Door Games & 88 & 66.72\\
\bottomrule
\end{longtable}
\endgroup{}

Pour identifier les meilleurs développeurs en fonction du score
Metacritic et du gain de joueurs actifs, nous avons exécuté une requête
SQL qui calcule la moyenne des scores Metacritic et des gains moyens des
joueurs actifs pour les jeux de chaque développeur, puis ordonné les
résultats pour mettre en évidence les meilleurs. En évaluant la moyenne
du gain, nous pouvons déduire si un jeu réussit à élargir sa base de
joueurs ou si, au contraire, il la perd sur la période étudiée.

Nous avons utilisé HAVING pour inclure uniquement les développeurs dont
les jeux ont une moyenne de score Metacritic supérieure à 75 et un gain
moyen de joueurs actifs positif, ce qui indique une performance
favorable à la fois des critiques et en termes d'acquisition de joueurs.
L'analyse montre des développeurs tels que ZA/UM et Beat Games avec des
scores Metacritic très élevés et des gains significatifs de joueurs
actifs, suggérant que leurs jeux sont non seulement bien reçus par les
critiques mais attirent également de nouveaux joueurs. Par exemple,
ZA/UM avec un score Metacritic moyen de 97 et un gain moyen de joueurs
de près de 40 dénote un succès écrasant sur les deux fronts.

Il est également intéressant de noter que les développeurs bien connus
et établis, tels que Beat Games et Relic Entertainment, apparaissent
avec de bons scores Metacritic et des gains élevés de joueurs actifs.
Cela peut indiquer que la notoriété de la marque et l'historique de
développement de jeux de qualité contribuent à la croissance continue de
la base de joueurs.

\newpage

\hypertarget{requuxeate-4-le-plusminus}{%
\subsection{Requête 4 : Le Plus/Minus}\label{requuxeate-4-le-plusminus}}

\begingroup\fontsize{10}{12}\selectfont

\begin{longtable}[t]{>{\raggedright\arraybackslash}p{2cm}r>{\raggedleft\arraybackslash}p{2cm}>{\raggedright\arraybackslash}p{0.5cm}>{\raggedleft\arraybackslash}p{3cm}>{\raggedleft\arraybackslash}p{3cm}r}
\caption{\label{tab:results-table plus/minus invisible}Requête 4, Le Plus/Minus}\\
\toprule
Jeu & Prix & Avis Metacritic & Mois & Moyenne des joueurs actifs & Moyenne des joueurs actifs de l'année dernière & +/-\\
\midrule
Counter-Strike: Global Offensive & 0 & 83 & April 2020 & 857604.2 & 351989.9 & 505614.3\\
Grand Theft Auto V & 0 & 96 & April 2016 & 31671.3 & 192714.0 & -161042.7\\
\bottomrule
\end{longtable}
\endgroup{}

La requête SQL a pour but d'analyser les variations du nombre de joueurs
actifs au cours des mois d'avril (mois choisi arbitrairement) pour des
jeux sélectionnés, basée sur des critères précis. La requête inclut des
colonnes clés qui sont critiques pour l'analyse :

\begin{enumerate}

\item AppID et Name pour identifier de manière unique et nommer le jeu.
\smallskip
\item Price pour observer l'influence du coût sur la participation des joueurs.
\smallskip
\item MetacriticScore pour filtrer les jeux selon leur accueil critique, ne retenant que ceux ayant un score de 70 ou plus.
\smallskip
\item AvgPlayers pour le nombre moyen de joueurs en avril de l'année courante, et
\smallskip
\item LastYearPlayers pour le nombre moyen de joueurs en avril de l'année précédente.
\smallskip
\end{enumerate}

Le calcul du PlusMinus est au coeur de la requête, elle estla différence
entre le nombre moyen de joueurs du mois d'avril courant (AvgPlayers) et
celui de l'année précédente (LastYearPlayers). Ce calcul révèle les
variations annuelles du nombre de joueurs, offrant un indicateur direct
de la croissance ou de la réduction de l'engagement des joueurs. Il y a
aussi un filtrage suplémentaire, seuls les jeux avec un score Metacritic
d'au moins 70 sont inclus, focalisant l'analyse sur des jeux bien reçus
par la critique. En plus, du fait que seuls les mois d'avril sont
considérés, ce qui permet une comparaison cohérente d'une année sur
l'autre.

Cette condition filtre le bruit potentiel que pourraient introduire des
jeux moins performants ou mal reçus, permettant une évaluation plus
précise de l'impact des critiques positives sur le succès d'un jeu. Les
variations dans le nombre de joueurs pour ces jeux bien notés peuvent
souvent refléter une réponse directe à des facteurs externes comme les
campagnes promotionnelles, les mises à jour de contenu significatives ou
les changements dans les tendances de consommation des médias.

Les résultats de cette analyse peuvent aider les éditeurs de jeux et les
développeurs à optimiser leurs stratégies de lancement et de promotion.
Par exemple, un jeu affichant une augmentation significative des joueurs
pourrait indiquer une réception très favorable à une récente mise à jour
ou à une modification de prix, tandis qu'une forte baisse pourrait
signaler des problèmes nécessitant une attention rapide, tels que des
problèmes techniques ou un contenu insatisfaisant.

Les insights tirés de cette analyse peuvent également guider les
décisions de planification à long terme. Comprendre les tendances sur
plusieurs années permet aux entreprises de mieux prédire les périodes de
haute activité et d'ajuster leurs ressources en conséquence, que ce soit
pour le support technique, le marketing ou le développement de contenu
supplémentaire. En outre, ces données fournissent une fenêtre sur
l'engagement des joueurs, offrant une mesure claire de la rétention et
de l'acquisition de joueurs au fil du temps. Ceci est particulièrement
pertinent dans l'industrie du jeu vidéo où l'engagement des joueurs est
un indicateur clé de succès continu. Les jeux qui réussissent à
maintenir ou à augmenter leur base de joueurs sur des périodes
prolongées peuvent souvent bénéficier de meilleures opportunités de
monétisation et d'une communauté plus solide.

Cette requête SQL offre des insights utiles sur l'évolution annuelle du
nombre de joueurs pour des jeux bien notés et peut partiellement
répondre à votre question sur l'influence des critiques et du prix
initial. Cependant, elle ne cible pas spécifiquement les premiers mois
après la sortie des jeux, ni n'analyse directement le lien entre le prix
initial et l'engagement des joueurs.

\newpage

\hypertarget{requuxeate-5-le-game-count}{%
\subsection{Requête 5 : Le Game
Count}\label{requuxeate-5-le-game-count}}

\begingroup\fontsize{10}{12}\selectfont

\begin{longtable}[t]{lrr}
\caption{\label{tab:results-table gamecount invisible}Requête 5, Le Gamecount}\\
\toprule
Catégorie du jeu & Nombre de jeu & Moyenne des joueurs actifs\\
\midrule
Note haute et bas prix & 201 & 9001.76\\
Autre & 562 & 5344.39\\
Note basse et haut prix & 187 & 3075.04\\
\bottomrule
\end{longtable}
\endgroup{}

Cette requête permet de classé chaque jeu en fonction de son score
MetacriticScore et de son prix. Les jeux avec un score Metacritic
supérieur ou égal à 70 et un prix inférieur ou égal à 20 euros sont
classés comme `Note haute et bas prix'. Les jeux avec un score
Metacritic inférieur à 70 et un prix supérieur à 20 euros sont classés
comme `Note basse et haut prix'. Tous les autres jeux sont classés comme
`Autre'. Pour chaque jeu, on calcule la moyenne des joueurs actifs. Elle
permet aussi de regrouper les jeux par AppID, ainsi que par leur score
Metacritic et leur prix, ce qui est essentiel pour obtenir une moyenne
précise des joueurs actifs pour chaque jeu unique. La requête compte le
nombre de jeux dans chaque catégorie et calcule la moyenne des joueurs
actifs par catégorie, en arrondissant cette moyenne à deux décimales.

La pertinence de cette requête réside dans sa capacité à fournir des
insights sur la popularité actuelle d'un jeu : En calculant le nombre
moyen de joueurs actifs, nous pouvons évaluer quels jeux captent et
maintiennent l'intérêt des joueurs sur la plateforme Steam. Cette
information est cruciale pour comprendre si un prix plus bas et des
critiques plus positives sont associés à une base de joueurs plus grande
et plus stable, ce qui est une donnée clé pour la stratégie d'un prix et
de marketing des éditeurs de jeux vidéo.

La catégorie `Note haute et bas prix' a le plus grand nombre de joueurs
actifs en moyenne, ce qui suggère que les jeux qui sont bien notés et
qui sont offerts à un prix inférieur ou égal à 20 euros ont tendance à
attirer et à maintenir une base de joueurs plus importante.

Les résultats suggèrent qu'un prix compétitif est important, surtout si
le jeu a reçu des évaluations positives. Les producteurs peuvent
envisager des stratégies de prix flexibles pour maximiser la base de
joueurs actifs.

Les jeux avec des scores Metacritic élevés tendent à attirer plus de
joueurs, ce qui met en lumière l'importance de viser une haute qualité
dans le développement pour obtenir de bonnes critiques. Il semble y
avoir un segment de marché distinct pour les jeux bien notés et bon
marché qui pourraient représenter un bon retour sur investissement en
termes de base de joueurs actifs.

L'analyse des jeux dans la catégorie `Autre' pourrait révéler des niches
ou des opportunités pour des jeux qui ne correspondent pas aux deux
principaux ensembles de critères mais qui ont quand même une base de
joueurs active significative.

\bigskip

En somme, pour les futurs développeurs ou producteurs de jeux, ces
informations sont cruciales car elles offrent des insights sur ce qui
peut influencer la popularité et la longévité d'un jeu sur le marché. En
se basant sur de telles données, ils peuvent affiner leurs stratégies de
développement, de marketing et de tarification pour améliorer les
chances de succès commercial de leurs jeux.

\normalsize

\hypertarget{matuxe9riel-et-muxe9thodes}{%
\chapter{Matériel et Méthodes}\label{matuxe9riel-et-muxe9thodes}}

\hypertarget{logiciels}{%
\section{Logiciels}\label{logiciels}}

Au cours de ce projet nous avons utiliser de nombreux logiciels :

\hypertarget{les-logiciels-de-communication-et-de-partage-des-donnuxe9es}{%
\subsection{Les logiciels de communication et de partage des
données}\label{les-logiciels-de-communication-et-de-partage-des-donnuxe9es}}

\begin{enumerate}
\def\labelenumi{\arabic{enumi}.}
\item
  WhatsApp \footnote{Version : 2.24.8.86.}: Pour planifer notre projet,
  nous avons commmencer par utiliser ce service de discussion
  instantanée.
\item
  Google Docs/Google Drive \footnote{Version 2.20.181.04.45.}: Nous
  avons par la suite essayer de nous partager nos avancées avec ce
  logiciel mais nous nous sommes vite rendus compte qu'il existait de
  meilleure option.
\item
  GitHub \footnote{Service en ligne, mis à jour continuellement.}: Nous
  avons ensuite choisi de continuer avec GitHub, le partage de code se
  faisant plus simplement.
\end{enumerate}

\bigskip

\hypertarget{les-logiciels-de-ruxe9daction}{%
\subsection{Les logiciels de
rédaction}\label{les-logiciels-de-ruxe9daction}}

\begin{enumerate}
\def\labelenumi{\arabic{enumi}.}
\item
  Overleaf \footnote{Service en ligne, mis à jour continuellement.}:
  Lors du début du projet, nous avions commencé à rédiger notre rapport
  sur Overleaf comme Mme.Bringay l'avait conseillé lors de ses cours.
  Overleaf nous a aussi permit de corriger notre texte.
\item
  Rstudio \footnote{RStudio : 2023.12.1.}: À la fin de la préparation du
  projet, nous avions toujours pas rédiger le projet en .rmd mais nous
  sommes arrivés tout de même à le rendre dans les délais en utilisant
  la rédaction du rapport en tex d'Overleaf. Tout en ajoutant de
  nouvelles choses comme le fait d'utiliser directement les bases de
  données de PhpMyAdmin dans le rapport en .rmd comme Mme.Demangeot nous
  l'avais recommandé.
\item
  ChatGPT \footnote{Mise à jour du 13 février, 2024.}: Son utilisation
  nous a facilité la rédaction de code sur RStudio lors du nettoyage des
  données mais aussi lors de ce rapport. Cela a permit de nous débloquer
  de situation où nous étions limité en terme de connaissance.
\item
  LanguageTool \footnote{Service en ligne, mis à jour continuellement.}:
  Cela nous a permis de corriger les éventuelles fautes d'orthographes
  et de syntaxes dans notre projet.
\end{enumerate}

\newpage

\hypertarget{les-logiciels-de-traitement-et-de-gestion-de-donnuxe9es}{%
\subsection{Les logiciels de traitement et de gestion de
données}\label{les-logiciels-de-traitement-et-de-gestion-de-donnuxe9es}}

\begin{enumerate}
\def\labelenumi{\arabic{enumi}.}
\item
  Excel/LibreOffice Calc \footnote{Microsoft Excel : Version 2403;
    LibreOffice Calc : Version 7.4.3.}: Nous avons visualisé les données
  avec ces logiciels et observé de nombreux problèmes de formatage.
\item
  Mocodo \footnote{Version 4.2.4.}: Cela nous a servi pour la
  modélisation conceptuelle de données.
\item
  Wamp/PhpMyAdmin \footnote{Wampserver : 3.3.2; PhpMyAdmin : 5.2.1.}:
  L'importation des données sur PhpMyAdmin permet la gestion des bases
  de données MySQL.
\item
  RStudio \footnote{RStudio : 2023.12.1.}: Nous a permis de nettoyer nos
  données et sélectionner les variables utiles.
\end{enumerate}

\hypertarget{ordinateur}{%
\section{Ordinateur}\label{ordinateur}}

\begin{enumerate}
\item VivoBook ASUS X515JA, Windows, Intel(R) Core(TM) i3-1005G1 CPU @ 1.20GHz, Système d’exploitation 64 bits 
\smallskip
\item Asus ROG FX503, Windows, Intel(R) Core(TM) i7-7700HQ CPU @ 2.80GHz, Système d’exploitation 64 bits
\end{enumerate}

\hypertarget{analyse-exploratoire-des-donnuxe9es}{%
\chapter{Analyse Exploratoire des
Données}\label{analyse-exploratoire-des-donnuxe9es}}

L'analyse exploratoire des données constitue une étape cruciale de notre
étude. Elle nous permet de comprendre les caractéristiques fondamentales
des données et d'identifier les tendances, anomalies ou patterns
éventuels. Cette compréhension initiale est essentielle pour orienter
nos analyses statistiques ultérieures et pour assurer l'application
correcte des méthodes statistiques.

\hypertarget{moduxe9lisation-de-la-variable-price}{%
\section{Modélisation de la variable
`Price'}\label{moduxe9lisation-de-la-variable-price}}

se trouve la méthode utilisée pour calculer les statistiques est robuste
et efficace pour notre analyse.

\begin{longtable}[]{@{}lr@{}}
\caption{Résumé des statistiques des prix des jeux}\tabularnewline
\toprule\noalign{}
Statistique & Valeur \\
\midrule\noalign{}
\endfirsthead
\toprule\noalign{}
Statistique & Valeur \\
\midrule\noalign{}
\endhead
\bottomrule\noalign{}
\endlastfoot
Moyenne & 20.587 \\
Médiane & 19.990 \\
1er Qu. & 4.490 \\
3ème Qu. & 29.990 \\
Min. & 0.000 \\
Max. & 69.990 \\
Mode & 0.000 \\
Variance & 307.458 \\
Ecart-type & 17.534 \\
Coefficient de variation & 85.170 \\
Skewness & 0.681 \\
Kurtosis & 2.749 \\
\end{longtable}

Pour visualiser la distribution des prix des jeux sur Steam, nous avons
opté pour un histogramme, qui est particulièrement adapté pour montrer
la fréquence des différentes gammes de prix dans notre jeu de données.
Ce graphique nous permet d'observer la concentration des prix autour de
certaines valeurs et d'identifier la présence de jeux gratuits, qui sont
représentés comme une modalité distincte.

\includegraphics{rapport-G17_files/figure-latex/histogramme invisible-1.pdf}
Cet histogramme montre que la majorité des jeux sont concentrés dans une
gamme de prix inférieure à 20€, avec une proportion significative de
jeux proposés gratuitement. Précisément, 22.7\% des jeux listés sont
gratuits, soulignant la popularité des modèles freemium ou des offres
promotionnelles initiales. En outre, environ 61.5\% des jeux coûtent
moins de 20€, ce qui reflète une stratégie de tarification compétitive
courante dans l'industrie pour attirer une base de joueurs plus large.

Les pics de distribution à 0€ et autour de 10€ à 20€ peuvent être
attribués à la popularité des jeux mobiles et indépendants, souvent
tarifés dans cette fourchette pour maximiser l'accessibilité tout en
garantissant une certaine rentabilité. Les jeux tarifés au-delà de 20€,
qui représentent une plus petite fraction du marché, sont généralement
des titres plus substantiels, offrant des expériences plus profondes ou
des contenus plus étendus, justifiant ainsi leur coût supérieur.

Cette distribution des prix a des implications directes tant pour les
développeurs que pour les consommateurs, influençant les décisions
d'achat des joueurs et les stratégies de tarification des éditeurs. Pour
les développeurs, comprendre ce schéma peut aider à positionner leurs
jeux de manière optimale dans un marché compétitif. Pour les
consommateurs, cela signifie une diversité de choix qui peut s'adapter à
divers budgets et préférences de jeu.

\newpage

\includegraphics{rapport-G17_files/figure-latex/nuage de point invisible-1.pdf}
\bigskip

L'analyse de ce nuage de points révèle que la majorité des jeux ayant un
score de presse supérieur à 90 ou un score situé entre 50 et 75
présentent un nombre moyen de joueurs par mois relativement faible, à
l'exception de quelques titres. En revanche, pour les jeux ayant un
score inférieur à 50, on observe une plus grande variabilité dans le
nombre moyen de joueurs par mois, bien que la plupart de ces jeux aient
un nombre moyen de joueurs/mois faible, certains d'entre eux
enregistrent un nombre relativement important, voire très important pour
une dizaine de jeux, se situant entre 250 000 et 1500 000.

L'échelle logarithmique en base 10, utilisée ici pour le nombre moyen de
joueurs par mois, permet de mieux discerner les différences entre les
jeux ayant de petits et de grands nombres de joueurs. Cette mise à
l'échelle est particulièrement utile pour ce jeu de données, car sans
elle, les quelques jeux avec des nombres extrêmement élevés de joueurs
pourraient éclipser la variabilité parmi les jeux moins populaires.
Grâce à l'échelle logarithmique, on peut observer que même parmi les
jeux avec des scores de presse moins élevés, certains peuvent quand même
attirer un nombre substantiel de joueurs, ce qui pourrait être masqué
avec une échelle linéaire standard.

En ce qui concerne les jeux ayant un score Metacritic compris entre 75
et 90, on remarque que leur nombre moyen de joueurs est généralement
plus élevé que dans les autres classes, le nombre moyen de joueurs pour
ces jeux se situe majoritairement entre 200 000 et 750 000, démontrant
ainsi une popularité plus marquée auprès des joueurs. \newpage

\includegraphics{rapport-G17_files/figure-latex/diagramme invisible-1.pdf}
\bigskip

L'analyse du diagramme en barres met en évidence des disparités dans le
nombre moyen de joueurs selon les catégories de jeux. Il apparaît
clairement que les jeux gratuits ayant un score de presse supérieur à 70
bénéficient d'un nombre moyen de joueurs considérablement plus élevé par
rapport aux autres catégories, bien que les jeux gratuits avec un score
de presse inférieur à 70 enregistrent également un nombre important de
joueurs, celui-ci reste inférieur à la première catégorie.

En ce qui concerne les jeux payants, on observe que ceux ayant un score
Metacritic supérieur à 70 enregistrent un nombre moyen de joueurs plus
important que les jeux dont le score est inférieur à 70. Cette tendance
suggère que les scores attribués par la presse peuvent avoir un impact
sur le nombre de joueurs moyen, en particulier pour les jeux payants.

En outre, les jeux gratuits et les jeux payants bien notés par la presse
semblent être les catégories les plus susceptibles d'attirer un nombre
important de joueurs. \newpage

\hypertarget{analyse-et-ruxe9sultats}{%
\chapter{Analyse et Résultats}\label{analyse-et-ruxe9sultats}}

\hypertarget{importance-des-moduxe8les-simples}{%
\section{Importance des modèles
simples}\label{importance-des-moduxe8les-simples}}

Comme mentionné dans les consignes de projet de ce chapitre, il est
crucial de commencer avec un modèle simple avant d'adopter des approches
plus complexes. Cette méthode permet une meilleure compréhension des
données et se révèle être davantage abordable pour notre niveau d'étude.
Le principe d'utiliser un modèle que l'on maîtrise assure une
interprétation correcte et fiable des résultats, facilitant ainsi les
décisions basées sur ces analyses.

\hypertarget{test-anova}{%
\section{Test ANOVA}\label{test-anova}}

Pour aller plus loin dans notre analyse statistique, il est essentiel de
comparer les moyennes des nombres de joueurs parmi les différentes
catégories de jeux que nous nous sommes crée. Pour se faire, nous avons
réalisé un test ANOVA (Analyse de Variance) sur les données des jeux à
notre disposition. Nous avons regroupé les jeux en fonction de leur
identifiant unique (AppID) et avons introduit une nouvelle variable
appelée ``Catégorie''. Cette variable indique si un jeu appartient à
l'une des six catégories que nous avons définies :

\begin{enumerate}
\item Jeux gratuits avec un score inférieur à 70,
\item Jeux gratuits avec un score supérieur à 70, 
\item Jeux payants dont le prix est inférieur à 20€ avec un score inférieur à 70,
\item Jeux payants dont le prix est inférieur à 20€ avec un score supérieur à 70,
\item Jeux payants dont le prix est supérieur à 20€ avec un score inférieur à 70,
\item Jeux payants dont le prix est supérieur à 20€ avec un score supérieur à 70.
\end {enumerate}
\bigskip

Le but de cette approche est de déterminer s'il existe une différence
significative dans le nombre moyen de joueurs entre les différentes
catégories de jeux. Pour effectuer ces tests ANOVA sur R, il est
recommandé d'utiliser la fonction aov().

\hypertarget{ruxe9sultats}{%
\section{Résultats}\label{ruxe9sultats}}

Le test ANOVA réalisé pour comparer les moyennes du nombre de joueurs
actifs entre différentes catégories de prix et de scores Metacritic a
montré des différences significatives. Comme indiqué dans la partie
statistique descriptive, nous observons que les jeux gratuits ou à bas
prix ayant une note élevée attirent et maintiennent un nombre plus élevé
de joueurs, ce qui est démontré par les résultats statistiquement
significatifs avec une valeur de Pr(\textgreater F) inférieure à 2e-16
(16((2 * 10 ˆ -16))).

\medskip

Ces résultats confirment l'hypothèse que le prix et la qualité perçue,
mesurée par les scores Metacriticscore, jouent un rôle crucial dans la
popularité des jeux sur Steam. Ces variables ont un impact significatif
sur le nombre de joueurs, soulignant l'importance pour les développeurs
de cibler ces aspects lors de la mise sur le marché de nouveaux jeux.

Les analyses ont montré que des tests statistiques plus avancés tels que
l'ANOVA peuvent fournir des éléments précieux pour notre étude.

Les implications de ces résultats sont significatives pour la stratégie
de développement et de marketing des jeux. Les développeurs doivent
considérer le positionnement de prix soigneusement tout en s'assurant
que la qualité du jeu est susceptible de recevoir des critiques
positives pour maximiser l'attraction et la rétention des joueurs. La
confirmation de ces effets par le test ANOVA renforce la nécessité d'une
approche équilibrée entre prix attractif et haute qualité perçue pour
réussir dans un marché aussi compétitif que celui des jeux sur Steam.

\hypertarget{conclusion}{%
\section{Conclusion}\label{conclusion}}

Ce chapitre a mis en lumière l'importance d'utiliser des modèles comme
celui de l'ANOVA, pour montrer comment des variables comme le prix et
les critiques impactent significativement le nombre de joueurs actifs au
sein d'une communanté. Ces découvertes doivent être intégrées dans les
phases de planification et de développement des jeux pour optimiser leur
succès commercial.

Il serait judicieux d'explorer davantage comment ces facteurs
interagissent avec d'autres variables telles que le genre du jeu, les
campagnes tarifaires promotionnelles, ou les mises à jour
post-lancement. Des analyses supplémentaires pourraient inclure des
modèles de régression multiple ou des techniques de machine learning
pour prédire plus précisément les tendances du marché et les
comportements des joueurs, permettant ainsi une adaptation plus fine des
stratégies de lancement et de promotion des jeux.

\hypertarget{discussion}{%
\chapter{Discussion}\label{discussion}}

Les résultats du test d'ANOVA ont mis en évidence que la variable
``catégorie'' avait un effet hautement significatif sur le nombre moyen
de joueurs, avec une valeur de Pr(\textgreater F) (qui représente la
valeur p associée à la statistique F et qui indique la significativité
des différences entre les groupes ou l'effet d'un facteur) inférieure à
2e-16((2 * 10 \^{} -16). Cette valeur étant bien inférieure au seuil de
significativité de 0,05, cela indique que les différences observées
entre les catégories de jeux ne sont pas dues au hasard et que le prix
et le score attribué par la presse à un jeu ont un impact important sur
le nombre moyen des joueurs par mois.

\hypertarget{synthuxe8se-des-ruxe9sultats}{%
\section{Synthèse des résultats}\label{synthuxe8se-des-ruxe9sultats}}

Dans le chapitre précédent, nous avons démontré à travers un test ANOVA
que les catégories de prix et les scores Metacritic ont un effet
significatif sur le nombre moyen de joueurs actifs. Les résultats
obtenus sont hautement significatifs, avec une valeur de
Pr(\textgreater F) inférieure à 2e-16, indiquant une influence claire de
ces facteurs sur la popularité des jeux sur Steam (Chapitre 5, section
5.3). Cette découverte est en adéquation avec les observations
préliminaires faites lors de l'analyse exploratoire des données où il a
été observé que les jeux gratuits ou à bas prix et bien notés attirent
et maintiennent un nombre plus élevé de joueurs (Chapitre 4).

\hypertarget{impact-des-prix-et-de-metacriticscore}{%
\section{Impact des prix et de
MetacriticScore}\label{impact-des-prix-et-de-metacriticscore}}

Les données indiquent que non seulement les jeux gratuits mais aussi
ceux vendus à un prix inférieur à 20 euros, s'ils sont accompagnés de
critiques positives, ont tendance à attirer un nombre plus élevé de
joueurs. Cette tendance souligne l'importance d'une tarification
stratégique combinée à la qualité perçue du jeu, comme l'ont montré les
catégories de jeux dans nos analyses, où les jeux avec un score
Metacritic supérieur à 70 enregistrent systématiquement un plus grand
nombre de joueurs, quelle que soit la tranche de prix (Chapitre 4,
Tableau des répartitions par catégorie de prix et note Metacritic).

\medskip

Cette dynamique est également visible dans le comportement des joueurs,
où les jeux ayant reçu des critiques favorables attirent non seulement
des joueurs initialement mais parviennent aussi à les retenir, ce qui
suggère une corrélation entre les scores de critiques et la durabilité
de l'engagement des joueurs. La discussion de ces résultats dans la
section des analyses exploratoires montre clairement cette tendance
(Chapitre 4).

\hypertarget{conclusion-1}{%
\section{Conclusion}\label{conclusion-1}}

La capacité à interpréter correctement l'impact des prix et des
critiques ouvre la voie à des décisions de développement et de marketing
plus informées. En utilisant ces informations, les développeurs peuvent
mieux positionner leurs jeux sur le marché, attirant efficacement et
durablement les joueurs.

\hypertarget{conclusion-et-perspectives}{%
\chapter{Conclusion et perspectives}\label{conclusion-et-perspectives}}

\hypertarget{conclusions-principales}{%
\section{Conclusions principales}\label{conclusions-principales}}

Notre étude a établi que le prix et les critiques des jeux ont une
influence significative sur le nombre de joueurs actifs sur Steam. Les
jeux bien notés et à bas prix tendent à attirer et à retenir un plus
grand nombre de joueurs, validant ainsi l'importance de ces deux
facteurs pour les développeurs et les marketeurs de jeux vidéo.

\hypertarget{considuxe9rations-pour-les-duxe9veloppeurs-de-jeux-viduxe9o}{%
\section{Considérations pour les développeurs de jeux
vidéo}\label{considuxe9rations-pour-les-duxe9veloppeurs-de-jeux-viduxe9o}}

Les développeurs doivent prendre en compte ces résultats pour optimiser
leurs stratégies de lancement. Placer un jeu à un prix compétitif tout
en s'assurant de la qualité susceptible d'attirer des critiques
positives peut être une stratégie gagnante. Les jeux bien évalués par
les critiques semblent bénéficier d'une base de joueurs plus large et
plus durable, ce qui est crucial pour la rentabilité à long terme au
sein de la communauté de Steam.

\hypertarget{perspectives}{%
\section{Perspectives}\label{perspectives}}

A court terme, il serait pertinent d'approfondir l'analyse pour explorer
les effets des différentes informations que possèdent les jeux et leur
impact sur l'engagement des joueurs. Sur le long terme, il serait
judicieux d'étudier l'impact des nouvelles technologies de jeu et des
modèles économiques émergents, tels que les jeux basés sur le cloud et
les abonnements, pour anticiper et influencer les tendances futures du
marché. L'intégration de ces perspectives dans la planification
stratégique et opérationnelle pourrait potentiellement transformer les
pratiques de développement de jeux, maximisant ainsi l'impact et le
succès des jeux sur des plateformes comme Steam.

\hypertarget{difficultuxe9s-rencontruxe9es}{%
\section{Difficultés rencontrées}\label{difficultuxe9s-rencontruxe9es}}

Partie Base de Données : \smallskip Le principale problème que nous
avons rencontré réside dans la recherche de jeu de données adéquats à
notre question de recherche. Nous avons changé de datasets plus de
quatre fois et modifié notre question de recherche à plusieurs reprises
à cause de la difficulté à trouver des données adéquates, au cours de
notre projet. Cette partie nous à sembler très limitante pour notre
créativité. Le partage des bases de données après le prétraitement
notamment, a posé problème, en particulier avant de choisir GitHub, qui
s'est révélé peu intuitif initialement. \medskip Partie Statistique :
\smallskip Nous n'avons pas rencontré de difficultés majeures pour le
nettoyage des données, car le dataset était déjà bien structuré.
Cependant, nous avons dû effectuer un tri minutieux des noms des jeux
pour éliminer les écritures en chinois simplifié et convertir les dates
en valeurs numériques, ce qui a pris beaucoup de temps. Cela aurait pu
etre très endommageant pour l'importation de nos dataset sur le serveur
de phpMyAdmin et effectuer la partie requete SQL. L'application des
notions de cours, notamment le codage en R Markdown, a été un défi, mais
notre dynamisme dans l'apprentissage nous a permis de surmonter ces
obstacles.

\hypertarget{synthuxe8se-et-engagement-du-groupe}{%
\section{Synthèse et engagement du
groupe}\label{synthuxe8se-et-engagement-du-groupe}}

Malgré les difficultés, l'engagement de chaque membre du groupe a été
essentiel pour surmonter les défis et produire un rapport final qui
reflète notre vision et répond à nos objectifs académiques. Cette
expérience a renforcé notre capacité de collaboration et d'utilisation
de diverses ressources pédagogiques, y compris des outils en ligne comme
ChatGPT, pour soutenir notre processus d'apprentissage.

En conclusion, ce projet a été une opportunité précieuse pour développer
nos compétences pratiques et théoriques en science des données.

\hypertarget{webographie}{%
\chapter*{Webographie}\label{webographie}}
\addcontentsline{toc}{chapter}{Webographie}

\hypertarget{refs}{}
\begin{CSLReferences}{0}{0}
\end{CSLReferences}

\bibliographystyle{elsarticle-harv}
\bibliography{references}

\begin{enumerate}
\def\labelenumi{\arabic{enumi}.}
\item
  OpenAI Blog sur ChatGPT - Un aperçu détaillé de ChatGPT, un modèle de
  langage avancé développé par OpenAI. \smallskip URL:
  \url{https://openai.com/blog/chatgpt} \medskip
\item
  SpringeR - Livre R - Un guide complet sur le logiciel statistique R,
  disponible sous forme de PDF. \smallskip URL:
  \url{https://biostatisticien.eu/springeR/livreR.pdf} \medskip
\item
  Kaggle Datasets - Une plateforme offrant un large éventail de datasets
  pour les projets de science des données. \smallskip URL:
  \url{https://www.kaggle.com/datasets} \medskip
\item
  LanguageTool - Un outil en ligne pour la correction grammaticale et
  orthographique. \smallskip URL: \url{https://languagetool.org/fr}
  \medskip
\item
  Overleaf Project - Une plateforme de rédaction et de collaboration en
  LaTeX. \smallskip URL: \url{https://www.overleaf.com/project} \medskip
\item
  Mocodo - Outil en ligne pour la modélisation conceptuelle de données.
  \smallskip URL: \url{https://www.mocodo.net/} \medskip
\item
  phpMyAdmin (Localhost) - Interface de gestion pour les bases de
  données MySQL, généralement utilisée localement. \smallskip URL:
  \url{http://localhost/phpmyadmin/} \medskip
\item
  datanovia - Outils en ligne pour aide statistique descriptive
  \smallskip URL: \url{https://www.datanovia.com/}
\end{enumerate}

\newpage

\hypertarget{annexes}{%
\chapter*{Annexes}\label{annexes}}
\addcontentsline{toc}{chapter}{Annexes}

\hypertarget{codes}{%
\section*{\texorpdfstring{\textbf{Codes}}{Codes}}\label{codes}}
\addcontentsline{toc}{section}{\textbf{Codes}}

\hypertarget{code-du-nettoyage-des-donnuxe9es}{%
\subsection*{Code du nettoyage des
données}\label{code-du-nettoyage-des-donnuxe9es}}
\addcontentsline{toc}{subsection}{Code du nettoyage des données}

\begin{Shaded}
\begin{Highlighting}[]
\CommentTok{\# Charger les bibliothèques}
\FunctionTok{library}\NormalTok{(dplyr) }
\CommentTok{\# Utilisée pour les manipulations de données.}
\FunctionTok{library}\NormalTok{(stringr) }
\CommentTok{\# Permet de travailler facilement avec des chaînes de caractères.}
\FunctionTok{library}\NormalTok{(lubridate) }
\end{Highlighting}
\end{Shaded}

\begin{verbatim}
## Warning: le package 'lubridate' a été compilé avec la version R 4.3.3
\end{verbatim}

\begin{Shaded}
\begin{Highlighting}[]
\CommentTok{\# Simplifie la gestion des dates.}

\CommentTok{\# Lire les fichiers}
\CommentTok{\# On charge les données depuis les fichiers CSV sur le disque local.}
\NormalTok{games }\OtherTok{\textless{}{-}} \FunctionTok{read.csv}\NormalTok{(}
  \StringTok{"C:}\SpecialCharTok{\textbackslash{}\textbackslash{}}\StringTok{Users}\SpecialCharTok{\textbackslash{}\textbackslash{}}\StringTok{antoc}\SpecialCharTok{\textbackslash{}\textbackslash{}}\StringTok{OneDrive}\SpecialCharTok{\textbackslash{}\textbackslash{}}\StringTok{Bureau}\SpecialCharTok{\textbackslash{}\textbackslash{}}\StringTok{Projet BD{-}SDD2}\SpecialCharTok{\textbackslash{}\textbackslash{}}\StringTok{games.csv"}\NormalTok{,}
  \AttributeTok{stringsAsFactors =} \ConstantTok{FALSE}\NormalTok{)}
\NormalTok{steam\_charts }\OtherTok{\textless{}{-}} \FunctionTok{read.csv}\NormalTok{(}
  \StringTok{"C:}\SpecialCharTok{\textbackslash{}\textbackslash{}}\StringTok{Users}\SpecialCharTok{\textbackslash{}\textbackslash{}}\StringTok{antoc}\SpecialCharTok{\textbackslash{}\textbackslash{}}\StringTok{OneDrive}\SpecialCharTok{\textbackslash{}\textbackslash{}}\StringTok{Bureau}\SpecialCharTok{\textbackslash{}\textbackslash{}}\StringTok{Projet BD{-}SDD2}\SpecialCharTok{\textbackslash{}\textbackslash{}}\StringTok{steam\_charts.csv"}\NormalTok{,}
  \AttributeTok{stringsAsFactors =} \ConstantTok{FALSE}\NormalTok{)}

\CommentTok{\# Renommer et préparer les colonnes}
\CommentTok{\# Ici, on renomme certaines colonnes pour simplifier}
\CommentTok{\# leur usage plus tard dans les scripts.}
\CommentTok{\# On a utilisé ChatGPT pour le code qui utilise la fonction mutate}
\NormalTok{steam\_charts }\OtherTok{\textless{}{-}}\NormalTok{ steam\_charts }\SpecialCharTok{\%\textgreater{}\%}
  \FunctionTok{rename}\NormalTok{(}\AttributeTok{AppID =}\NormalTok{ App.ID, }\AttributeTok{Name =}\NormalTok{ Game, }\StringTok{\textasciigrave{}}\AttributeTok{\%Gain}\StringTok{\textasciigrave{}} \OtherTok{=}\NormalTok{ X..Gain) }\SpecialCharTok{\%\textgreater{}\%}
  \FunctionTok{mutate}\NormalTok{(}\AttributeTok{Gain =} \FunctionTok{replace}\NormalTok{(Gain, Gain }\SpecialCharTok{==} \StringTok{"{-}"}\NormalTok{, }\DecValTok{0}\NormalTok{), }
         \CommentTok{\# Remplacer les tirets par des zéros pour éviter les erreurs de type.}
         \StringTok{\textasciigrave{}}\AttributeTok{\%Gain}\StringTok{\textasciigrave{}} \OtherTok{=} \FunctionTok{replace}\NormalTok{(}\StringTok{\textasciigrave{}}\AttributeTok{\%Gain}\StringTok{\textasciigrave{}}\NormalTok{, }\StringTok{\textasciigrave{}}\AttributeTok{\%Gain}\StringTok{\textasciigrave{}} \SpecialCharTok{==} \StringTok{"{-}"}\NormalTok{, }\DecValTok{0}\NormalTok{)) }\SpecialCharTok{\%\textgreater{}\%}
  \FunctionTok{mutate}\NormalTok{(}
    \AttributeTok{Gain =} \FunctionTok{as.numeric}\NormalTok{(}\FunctionTok{gsub}\NormalTok{(}\StringTok{"\%"}\NormalTok{, }\StringTok{""}\NormalTok{, }\FunctionTok{as.character}\NormalTok{(Gain), }\AttributeTok{fixed =} \ConstantTok{TRUE}\NormalTok{)),  }
    \CommentTok{\# Convertir les gains en numérique en supprimant le signe \%.}
    \StringTok{\textasciigrave{}}\AttributeTok{\%Gain}\StringTok{\textasciigrave{}} \OtherTok{=} \FunctionTok{as.numeric}\NormalTok{(}\FunctionTok{gsub}\NormalTok{(}\StringTok{"\%"}\NormalTok{, }\StringTok{""}\NormalTok{, }\FunctionTok{as.character}\NormalTok{(}\StringTok{\textasciigrave{}}\AttributeTok{\%Gain}\StringTok{\textasciigrave{}}\NormalTok{), }\AttributeTok{fixed =} \ConstantTok{TRUE}\NormalTok{))}
\NormalTok{  )}

\CommentTok{\# Filtrer les jeux}
\CommentTok{\# On élimine les jeux en chinois simplifié et }
\CommentTok{\# on ne garde que ceux présents dans steam\_charts.}
\CommentTok{\# On a utilisé ChatGPT pour le codes qui modifie le format de la date}

\NormalTok{games\_filtered }\OtherTok{\textless{}{-}}\NormalTok{ games }\SpecialCharTok{\%\textgreater{}\%}
  \FunctionTok{rename}\NormalTok{(}\AttributeTok{ReleaseDate =} \StringTok{\textasciigrave{}}\AttributeTok{Release.date}\StringTok{\textasciigrave{}}\NormalTok{) }\SpecialCharTok{\%\textgreater{}\%}
  \FunctionTok{filter}\NormalTok{(AppID }\SpecialCharTok{\%in\%}\NormalTok{ steam\_charts}\SpecialCharTok{$}\NormalTok{AppID) }\SpecialCharTok{\%\textgreater{}\%}
  \FunctionTok{filter}\NormalTok{(Supported.languages }\SpecialCharTok{!=} \StringTok{"[\textquotesingle{}Simplified Chinese\textquotesingle{}]"}\NormalTok{) }\SpecialCharTok{\%\textgreater{}\%}
  \FunctionTok{mutate}\NormalTok{(}\AttributeTok{ReleaseDate =} \FunctionTok{format}\NormalTok{(}\FunctionTok{mdy}\NormalTok{(ReleaseDate), }\StringTok{"\%Y/\%m/\%d"}\NormalTok{))  }
\CommentTok{\# Formatage de la date.}

\CommentTok{\# Nettoyage supplémentaire pour aligner les jeux et les données Steam}
\NormalTok{steam\_charts\_filtered }\OtherTok{\textless{}{-}}\NormalTok{ steam\_charts }\SpecialCharTok{\%\textgreater{}\%}
  \FunctionTok{filter}\NormalTok{(AppID }\SpecialCharTok{\%in\%}\NormalTok{ games\_filtered}\SpecialCharTok{$}\NormalTok{AppID)}

\CommentTok{\# Préparation finale des données de jeux}
\NormalTok{filtered\_games\_cleaned }\OtherTok{\textless{}{-}}\NormalTok{ games\_filtered }\SpecialCharTok{\%\textgreater{}\%}
  \FunctionTok{select}\NormalTok{(}
    \SpecialCharTok{{-}}\NormalTok{Estimated.owners, }\SpecialCharTok{{-}}\NormalTok{Peak.CCU, }\SpecialCharTok{{-}}\NormalTok{Required.age, }\SpecialCharTok{{-}}\NormalTok{DLC.count, }
    \SpecialCharTok{{-}}\NormalTok{About.the.game, }
    \SpecialCharTok{{-}}\NormalTok{Supported.languages, }\SpecialCharTok{{-}}\NormalTok{Full.audio.languages, }\SpecialCharTok{{-}}\NormalTok{Reviews, }\SpecialCharTok{{-}}\NormalTok{Header.image, }
    \SpecialCharTok{{-}}\NormalTok{Website,}
    \SpecialCharTok{{-}}\NormalTok{Support.url, }\SpecialCharTok{{-}}\NormalTok{Support.email, }\SpecialCharTok{{-}}\NormalTok{Windows, }\SpecialCharTok{{-}}\NormalTok{Mac, }\SpecialCharTok{{-}}\NormalTok{Linux,}
    \SpecialCharTok{{-}}\NormalTok{Metacritic.url, }\SpecialCharTok{{-}}\NormalTok{User.score, }\SpecialCharTok{{-}}\NormalTok{Score.rank, }\SpecialCharTok{{-}}\NormalTok{Achievements, }
    \SpecialCharTok{{-}}\NormalTok{Recommendations,}
    \SpecialCharTok{{-}}\NormalTok{Notes, }\SpecialCharTok{{-}}\NormalTok{Average.playtime.two.weeks, }\SpecialCharTok{{-}}\NormalTok{Median.playtime.forever,}
    \SpecialCharTok{{-}}\NormalTok{Median.playtime.two.weeks, }\SpecialCharTok{{-}}\NormalTok{Publishers, }\SpecialCharTok{{-}}\NormalTok{Categories, }
    \SpecialCharTok{{-}}\NormalTok{Genres, }\SpecialCharTok{{-}}\NormalTok{Tags,}
    \SpecialCharTok{{-}}\NormalTok{Screenshots, }\SpecialCharTok{{-}}\NormalTok{Movies}
\NormalTok{  ) }\SpecialCharTok{\%\textgreater{}\%}
  \FunctionTok{arrange}\NormalTok{(AppID) }
\CommentTok{\# Tri des données par AppID pour une meilleure organisation.}

\CommentTok{\# Sauvegarde des données nettoyées}
\FunctionTok{write.table}\NormalTok{(filtered\_games\_cleaned, }
\StringTok{"C:}\SpecialCharTok{\textbackslash{}\textbackslash{}}\StringTok{Users}\SpecialCharTok{\textbackslash{}\textbackslash{}}\StringTok{antoc}\SpecialCharTok{\textbackslash{}\textbackslash{}}\StringTok{OneDrive}\SpecialCharTok{\textbackslash{}\textbackslash{}}\StringTok{Bureau}\SpecialCharTok{\textbackslash{}\textbackslash{}}\StringTok{Projet BD{-}SDD2}\SpecialCharTok{\textbackslash{}\textbackslash{}}\StringTok{games\_clean.csv"}\NormalTok{, }
    \AttributeTok{sep =} \StringTok{";"}\NormalTok{, }\AttributeTok{row.names =} \ConstantTok{FALSE}\NormalTok{, }\AttributeTok{col.names =} \ConstantTok{TRUE}\NormalTok{, }
    \AttributeTok{fileEncoding =} \StringTok{"UTF{-}8"}\NormalTok{, }\AttributeTok{quote =} \ConstantTok{TRUE}\NormalTok{)}

\CommentTok{\# Préparation et sauvegarde finale des données Steam Charts}
\NormalTok{steam\_charts\_cleaned }\OtherTok{\textless{}{-}}\NormalTok{ steam\_charts\_filtered }\SpecialCharTok{\%\textgreater{}\%}
  \FunctionTok{select}\NormalTok{(AppID, }\SpecialCharTok{{-}}\NormalTok{Name, }\FunctionTok{everything}\NormalTok{()) }\SpecialCharTok{\%\textgreater{}\%}
  \FunctionTok{arrange}\NormalTok{(AppID)}

\FunctionTok{write.table}\NormalTok{(steam\_charts\_cleaned, }
\StringTok{"C:}\SpecialCharTok{\textbackslash{}\textbackslash{}}\StringTok{Users}\SpecialCharTok{\textbackslash{}\textbackslash{}}\StringTok{antoc}\SpecialCharTok{\textbackslash{}\textbackslash{}}\StringTok{OneDrive}\SpecialCharTok{\textbackslash{}\textbackslash{}}\StringTok{Bureau}\SpecialCharTok{\textbackslash{}\textbackslash{}}\StringTok{Projet BD{-}SDD2}\SpecialCharTok{\textbackslash{}\textbackslash{}}\StringTok{steam\_charts\_clean.csv"}\NormalTok{,}
            \AttributeTok{sep =} \StringTok{";"}\NormalTok{, }\AttributeTok{row.names =} \ConstantTok{FALSE}\NormalTok{, }\AttributeTok{col.names =} \ConstantTok{TRUE}\NormalTok{,}
            \AttributeTok{fileEncoding =} \StringTok{"UTF{-}8"}\NormalTok{, }\AttributeTok{quote =} \ConstantTok{TRUE}\NormalTok{)}
\end{Highlighting}
\end{Shaded}

\hypertarget{code-de-la-requuxeate-1}{%
\subsection*{Code de la requête 1}\label{code-de-la-requuxeate-1}}
\addcontentsline{toc}{subsection}{Code de la requête 1}

\begin{Shaded}
\begin{Highlighting}[]
\NormalTok{query }\OtherTok{\textless{}{-}} \StringTok{"}
\StringTok{SELECT ROUND(AVG(Price), 2) as \textquotesingle{}Price\textquotesingle{},}
\StringTok{       ROUND(AVG(MetacriticScore), 2) as \textquotesingle{}MetacriticScore\textquotesingle{}}
\StringTok{FROM jeu;}
\StringTok{"}
\NormalTok{results\_moy }\OtherTok{\textless{}{-}} \FunctionTok{dbGetQuery}\NormalTok{(con, query)}
\end{Highlighting}
\end{Shaded}

\begin{Shaded}
\begin{Highlighting}[]
\NormalTok{results\_moy }\OtherTok{\textless{}{-}}\NormalTok{ results\_moy }\SpecialCharTok{\%\textgreater{}\%}
  \FunctionTok{rename}\NormalTok{(}
    \StringTok{\textasciigrave{}}\AttributeTok{Moyenne des prix}\StringTok{\textasciigrave{}} \OtherTok{=}\NormalTok{ Price,}
    \StringTok{\textasciigrave{}}\AttributeTok{Moyenne des notes Metacritic}\StringTok{\textasciigrave{}} \OtherTok{=}\NormalTok{ MetacriticScore}
\NormalTok{  )}
\FunctionTok{kable}\NormalTok{(results\_moy, }\StringTok{"latex"}\NormalTok{, }\AttributeTok{booktabs =} \ConstantTok{TRUE}\NormalTok{, }\AttributeTok{longtable =} \ConstantTok{TRUE}\NormalTok{, }
      \AttributeTok{caption=} \StringTok{"Requête 1, Les moyennes"}\NormalTok{ ) }\SpecialCharTok{\%\textgreater{}\%}
  \FunctionTok{kable\_styling}\NormalTok{(}\AttributeTok{font\_size =} \DecValTok{10}\NormalTok{) }\SpecialCharTok{\%\textgreater{}\%}  
  \FunctionTok{column\_spec}\NormalTok{(}\DecValTok{1}\NormalTok{, }\AttributeTok{width =} \StringTok{"3cm"}\NormalTok{)}
\end{Highlighting}
\end{Shaded}

\begingroup\fontsize{10}{12}\selectfont

\begin{longtable}[t]{>{\raggedleft\arraybackslash}p{3cm}r}
\caption{\label{tab:results-table moyenne}Requête 1, Les moyennes}\\
\toprule
Moyenne des prix & Moyenne des notes Metacritic\\
\midrule
20.59 & 35.19\\
\bottomrule
\end{longtable}
\endgroup{}

\hypertarget{code-de-la-requuxeate-2}{%
\subsection*{Code de la requête 2}\label{code-de-la-requuxeate-2}}
\addcontentsline{toc}{subsection}{Code de la requête 2}

\begin{Shaded}
\begin{Highlighting}[]
\NormalTok{query }\OtherTok{\textless{}{-}} \StringTok{"}
\StringTok{SELECT }
\StringTok{  CASE }
\StringTok{    WHEN j.Price = 0 THEN \textquotesingle{}Gratuit\textquotesingle{}}
\StringTok{    WHEN j.Price \textless{} 20 THEN \textquotesingle{}Inférieur à 20\textquotesingle{}}
\StringTok{    ELSE \textquotesingle{}Supérieur à 20\textquotesingle{}}
\StringTok{  END AS PriceCategory,}
\StringTok{  ROUND(AVG(s.AvgPlayers), 2) as AvgActivePlayers }
\StringTok{FROM jeu j}
\StringTok{JOIN statistique s ON j.AppID = s.AppID}
\StringTok{GROUP BY PriceCategory}
\StringTok{ORDER BY CASE }
\StringTok{           WHEN PriceCategory = \textquotesingle{}Gratuit\textquotesingle{} THEN 1}
\StringTok{           WHEN PriceCategory = \textquotesingle{}Inférieur à 20\textquotesingle{} THEN 2}
\StringTok{           ELSE 3}
\StringTok{         END;}
\StringTok{"}
\NormalTok{results\_pri }\OtherTok{\textless{}{-}} \FunctionTok{dbGetQuery}\NormalTok{(con, query)}
\end{Highlighting}
\end{Shaded}

\begin{Shaded}
\begin{Highlighting}[]
\NormalTok{results\_pri }\OtherTok{\textless{}{-}}\NormalTok{ results\_pri }\SpecialCharTok{\%\textgreater{}\%}
  \FunctionTok{rename}\NormalTok{(}
    \StringTok{"Catégorie de Prix"} \OtherTok{=}\NormalTok{ PriceCategory,}
    \StringTok{"Moyenne des joueurs actifs"} \OtherTok{=}\NormalTok{ AvgActivePlayers}
\NormalTok{  )}
\FunctionTok{kable}\NormalTok{(results\_pri, }\StringTok{"latex"}\NormalTok{, }\AttributeTok{booktabs =} \ConstantTok{TRUE}\NormalTok{, }\AttributeTok{longtable =} \ConstantTok{TRUE}\NormalTok{, }
      \AttributeTok{caption=} \StringTok{"Requête 2, Les catégories des prix"}\NormalTok{ ) }\SpecialCharTok{\%\textgreater{}\%}
  \FunctionTok{kable\_styling}\NormalTok{(}\AttributeTok{font\_size =} \DecValTok{10}\NormalTok{) }\SpecialCharTok{\%\textgreater{}\%}  
  \FunctionTok{column\_spec}\NormalTok{(}\DecValTok{1}\NormalTok{, }\AttributeTok{width =} \StringTok{"4cm"}\NormalTok{)}
\end{Highlighting}
\end{Shaded}

\begingroup\fontsize{10}{12}\selectfont

\begin{longtable}[t]{>{\raggedright\arraybackslash}p{4cm}r}
\caption{\label{tab:results-table catégorie prix}Requête 2, Les catégories des prix}\\
\toprule
Catégorie de Prix & Moyenne des joueurs actifs\\
\midrule
Gratuit & 16146.02\\
Inférieur à 20 & 2512.70\\
Supérieur à 20 & 3625.50\\
\bottomrule
\end{longtable}
\endgroup{}

\hypertarget{code-de-la-requuxeate-3}{%
\subsection*{Code de la requête 3}\label{code-de-la-requuxeate-3}}
\addcontentsline{toc}{subsection}{Code de la requête 3}

\begingroup\fontsize{10}{12}\selectfont

\begin{longtable}[t]{>{\raggedright\arraybackslash}p{5cm}rr}
\caption{\label{tab:results-table developpeurs}Requête 3, Les developpeurs}\\
\toprule
Développeur & Moyenne des notes Metacritic & Moyenne des gains de joueurs\\
\midrule
ZA/UM & 97 & 39.95\\
Beat Games & 93 & 18.07\\
Relic Entertainment & 93 & 4.23\\
Wube Software LTD. & 90 & 115.09\\
Firaxis Games & 90 & 5.70\\
\addlinespace
ConcernedApe & 89 & 403.67\\
Mega Crit Games & 89 & 136.00\\
Motion Twin & 89 & 36.72\\
SkyBox Labs,Big Huge Games & 89 & 1.19\\
Cellar Door Games & 88 & 66.72\\
\bottomrule
\end{longtable}
\endgroup{}

\hypertarget{code-de-la-requuxeate-4}{%
\subsection*{Code de la requête 4}\label{code-de-la-requuxeate-4}}
\addcontentsline{toc}{subsection}{Code de la requête 4}

\begin{Shaded}
\begin{Highlighting}[]
\NormalTok{query }\OtherTok{\textless{}{-}} \StringTok{"}
\StringTok{WITH PlayerDifferences AS (}
\StringTok{    SELECT }
\StringTok{        j.AppID,}
\StringTok{        j.Name,}
\StringTok{        j.Price,                  }
\StringTok{        j.MetacriticScore,         }
\StringTok{        s.Month,}
\StringTok{        ROUND(s.AvgPlayers, 2) AS AvgPlayers,}
\StringTok{        LAG(ROUND(s.AvgPlayers, 2)) OVER (}
\StringTok{        PARTITION BY j.AppID ORDER BY s.Month) AS LastYearPlayers,}
\StringTok{        ROUND((s.AvgPlayers {-} LAG(s.AvgPlayers) OVER (}
\StringTok{        PARTITION BY j.AppID ORDER BY s.Month)), 2) AS PlusMinus}
\StringTok{    FROM }
\StringTok{        jeu j}
\StringTok{    JOIN }
\StringTok{        statistique s ON j.AppID = s.AppID}
\StringTok{    WHERE }
\StringTok{        s.Month LIKE \textquotesingle{}April\%\textquotesingle{} AND}
\StringTok{        j.MetacriticScore \textgreater{}= 70  }
\StringTok{)}

\StringTok{SELECT }
\StringTok{    Name,}
\StringTok{    Price,                     }
\StringTok{    MetacriticScore,           }
\StringTok{    Month,}
\StringTok{    AvgPlayers,}
\StringTok{    LastYearPlayers,}
\StringTok{    PlusMinus}
\StringTok{FROM }
\StringTok{    PlayerDifferences}
\StringTok{WHERE }
\StringTok{    PlusMinus = (SELECT MAX(PlusMinus) FROM PlayerDifferences)}
\StringTok{    OR PlusMinus = (SELECT MIN(PlusMinus) FROM PlayerDifferences)}
\StringTok{ORDER BY }
\StringTok{    PlusMinus DESC;}
\StringTok{"}
\NormalTok{results\_plusminus }\OtherTok{\textless{}{-}} \FunctionTok{dbGetQuery}\NormalTok{(con, query)}
\end{Highlighting}
\end{Shaded}

\begin{Shaded}
\begin{Highlighting}[]
\NormalTok{results\_plusminus }\OtherTok{\textless{}{-}}\NormalTok{ results\_plusminus }\SpecialCharTok{\%\textgreater{}\%}
  \FunctionTok{rename}\NormalTok{(}
    \StringTok{"Jeu"} \OtherTok{=}\NormalTok{ Name,}
    \StringTok{"Prix"} \OtherTok{=}\NormalTok{ Price,}
    \StringTok{"Avis Metacritic"}\OtherTok{=}\NormalTok{ MetacriticScore,}
    \StringTok{"Mois"}\OtherTok{=}\NormalTok{ Month,}
    \StringTok{"Moyenne des joueurs actifs"} \OtherTok{=}\NormalTok{ AvgPlayers,}
    \StringTok{"Moyenne des joueurs actifs de l\textquotesingle{}année dernière"} \OtherTok{=}\NormalTok{ LastYearPlayers,}
    \StringTok{"+/{-}"} \OtherTok{=}\NormalTok{ PlusMinus}
\NormalTok{  )}
\FunctionTok{kable}\NormalTok{(results\_plusminus, }\StringTok{"latex"}\NormalTok{, }\AttributeTok{booktabs =} \ConstantTok{TRUE}\NormalTok{, }\AttributeTok{longtable =} \ConstantTok{TRUE}\NormalTok{, }
      \AttributeTok{caption=} \StringTok{"Requête 4, Le Plus/Minus"}\NormalTok{ ) }\SpecialCharTok{\%\textgreater{}\%}
  \FunctionTok{kable\_styling}\NormalTok{(}\AttributeTok{font\_size =} \DecValTok{10}\NormalTok{) }\SpecialCharTok{\%\textgreater{}\%}  
  \FunctionTok{column\_spec}\NormalTok{(}\DecValTok{1}\NormalTok{, }\AttributeTok{width =} \StringTok{"2cm"}\NormalTok{)}\SpecialCharTok{\%\textgreater{}\%}
  \FunctionTok{column\_spec}\NormalTok{(}\DecValTok{3}\NormalTok{, }\AttributeTok{width =} \StringTok{"2cm"}\NormalTok{)}\SpecialCharTok{\%\textgreater{}\%}
  \FunctionTok{column\_spec}\NormalTok{(}\DecValTok{4}\NormalTok{, }\AttributeTok{width =} \StringTok{"0.5cm"}\NormalTok{)}\SpecialCharTok{\%\textgreater{}\%}
  \FunctionTok{column\_spec}\NormalTok{(}\DecValTok{5}\NormalTok{, }\AttributeTok{width =} \StringTok{"3cm"}\NormalTok{)}\SpecialCharTok{\%\textgreater{}\%}
  \FunctionTok{column\_spec}\NormalTok{(}\DecValTok{6}\NormalTok{, }\AttributeTok{width =} \StringTok{"3cm"}\NormalTok{)}
\end{Highlighting}
\end{Shaded}

\begingroup\fontsize{10}{12}\selectfont

\begin{longtable}[t]{>{\raggedright\arraybackslash}p{2cm}r>{\raggedleft\arraybackslash}p{2cm}>{\raggedright\arraybackslash}p{0.5cm}>{\raggedleft\arraybackslash}p{3cm}>{\raggedleft\arraybackslash}p{3cm}r}
\caption{\label{tab:results-table plus/minus}Requête 4, Le Plus/Minus}\\
\toprule
Jeu & Prix & Avis Metacritic & Mois & Moyenne des joueurs actifs & Moyenne des joueurs actifs de l'année dernière & +/-\\
\midrule
Counter-Strike: Global Offensive & 0 & 83 & April 2020 & 857604.2 & 351989.9 & 505614.3\\
Grand Theft Auto V & 0 & 96 & April 2016 & 31671.3 & 192714.0 & -161042.7\\
\bottomrule
\end{longtable}
\endgroup{}

\hypertarget{code-de-la-requuxeate-5}{%
\subsection*{Code de la requête 5}\label{code-de-la-requuxeate-5}}
\addcontentsline{toc}{subsection}{Code de la requête 5}

\begin{Shaded}
\begin{Highlighting}[]
\NormalTok{query }\OtherTok{\textless{}{-}} \StringTok{"}
\StringTok{WITH ScoredGames AS (}
\StringTok{  SELECT}
\StringTok{    j.AppID,}
\StringTok{    ROUND(AVG(s.AvgPlayers), 2) AS AvgActivePlayers,}
\StringTok{    CASE}
\StringTok{      WHEN j.MetacriticScore \textgreater{}= 70 }
\StringTok{        AND j.Price \textless{}= 20 THEN \textquotesingle{}Note haute et bas prix\textquotesingle{}}
\StringTok{      WHEN j.MetacriticScore \textless{} 70 }
\StringTok{        AND j.Price \textgreater{} 20 THEN \textquotesingle{}Note basse et haut prix\textquotesingle{}}
\StringTok{      ELSE \textquotesingle{}Autre\textquotesingle{}}
\StringTok{    END AS Category}
\StringTok{  FROM jeu j}
\StringTok{  INNER JOIN statistique s ON j.AppID = s.AppID}
\StringTok{  GROUP BY j.AppID, j.MetacriticScore, j.Price}
\StringTok{)}
\StringTok{SELECT}
\StringTok{  Category,}
\StringTok{  COUNT(AppID) AS GameCount,}
\StringTok{  ROUND(AVG(AvgActivePlayers), 2) AS AvgActivePlayers}
\StringTok{FROM ScoredGames}
\StringTok{GROUP BY Category;}
\StringTok{"}
\NormalTok{results\_gamecount }\OtherTok{\textless{}{-}} \FunctionTok{dbGetQuery}\NormalTok{(con, query)}
\end{Highlighting}
\end{Shaded}

\begin{Shaded}
\begin{Highlighting}[]
\NormalTok{results\_gamecount }\OtherTok{\textless{}{-}}\NormalTok{ results\_gamecount }\SpecialCharTok{\%\textgreater{}\%}
  \FunctionTok{rename}\NormalTok{(}
    \StringTok{"Catégorie du jeu"} \OtherTok{=}\NormalTok{ Category,}
    \StringTok{"Nombre de jeu"} \OtherTok{=}\NormalTok{ GameCount,}
    \StringTok{"Moyenne des joueurs actifs"}\OtherTok{=}\NormalTok{ AvgActivePlayers}
\NormalTok{  )}
\FunctionTok{kable}\NormalTok{(results\_gamecount, }\StringTok{"latex"}\NormalTok{, }\AttributeTok{booktabs =} \ConstantTok{TRUE}\NormalTok{, }\AttributeTok{longtable =} \ConstantTok{TRUE}\NormalTok{, }
      \AttributeTok{caption=} \StringTok{"Requête 5, Le Gamecount"}\NormalTok{ ) }\SpecialCharTok{\%\textgreater{}\%}
  \FunctionTok{kable\_styling}\NormalTok{(}\AttributeTok{font\_size =} \DecValTok{10}\NormalTok{)}
\end{Highlighting}
\end{Shaded}

\begingroup\fontsize{10}{12}\selectfont

\begin{longtable}[t]{lrr}
\caption{\label{tab:results-table gamecount2}Requête 5, Le Gamecount}\\
\toprule
Catégorie du jeu & Nombre de jeu & Moyenne des joueurs actifs\\
\midrule
Note haute et bas prix & 201 & 9001.76\\
Autre & 562 & 5344.39\\
Note basse et haut prix & 187 & 3075.04\\
\bottomrule
\end{longtable}
\endgroup{}

\hypertarget{code-du-la-moduxe9lisation-de-price}{%
\subsection*{Code du la modélisation de
`price'}\label{code-du-la-moduxe9lisation-de-price}}
\addcontentsline{toc}{subsection}{Code du la modélisation de `price'}

\begin{Shaded}
\begin{Highlighting}[]
\FunctionTok{library}\NormalTok{(moments)}
\FunctionTok{library}\NormalTok{(magrittr)}

\CommentTok{\# Données}
\NormalTok{jeu }\OtherTok{\textless{}{-}} \FunctionTok{dbReadTable}\NormalTok{(con, }\StringTok{"jeu"}\NormalTok{)}
\NormalTok{jeu\_price }\OtherTok{\textless{}{-}}\NormalTok{ jeu}\SpecialCharTok{$}\NormalTok{Price}

\CommentTok{\# Calcul des statistiques}
\NormalTok{table\_price }\OtherTok{\textless{}{-}} \FunctionTok{table}\NormalTok{(jeu\_price)}

\NormalTok{mode\_price }\OtherTok{\textless{}{-}} \FunctionTok{as.numeric}\NormalTok{(}\FunctionTok{names}\NormalTok{(}\FunctionTok{which.max}\NormalTok{(}\FunctionTok{table}\NormalTok{(jeu\_price))))}

\NormalTok{variance\_price }\OtherTok{\textless{}{-}} \FunctionTok{round}\NormalTok{(}\FunctionTok{var}\NormalTok{(jeu\_price) }\SpecialCharTok{*}\NormalTok{ (}\FunctionTok{length}\NormalTok{(jeu\_price) }\SpecialCharTok{{-}} \DecValTok{1}\NormalTok{) }
                        \SpecialCharTok{/} \FunctionTok{length}\NormalTok{(jeu\_price), }\DecValTok{3}\NormalTok{)}
\NormalTok{ecart\_type\_price }\OtherTok{\textless{}{-}} \FunctionTok{round}\NormalTok{(}\FunctionTok{sqrt}\NormalTok{(variance\_price), }\DecValTok{3}\NormalTok{)}
\NormalTok{mean\_price }\OtherTok{\textless{}{-}} \FunctionTok{mean}\NormalTok{(jeu\_price)}
\NormalTok{co\_price }\OtherTok{\textless{}{-}} \FunctionTok{round}\NormalTok{((ecart\_type\_price }\SpecialCharTok{/}\NormalTok{ mean\_price) }\SpecialCharTok{*} \DecValTok{100}\NormalTok{, }\DecValTok{3}\NormalTok{)}
\NormalTok{skewness\_price }\OtherTok{\textless{}{-}} \FunctionTok{round}\NormalTok{(}\FunctionTok{skewness}\NormalTok{(jeu\_price), }\DecValTok{3}\NormalTok{)}
\NormalTok{kurtosis\_price }\OtherTok{\textless{}{-}} \FunctionTok{round}\NormalTok{(}\FunctionTok{kurtosis}\NormalTok{(jeu\_price), }\DecValTok{3}\NormalTok{)}
\NormalTok{summary\_price }\OtherTok{\textless{}{-}} \FunctionTok{summary}\NormalTok{(jeu\_price)}

\CommentTok{\#Utilisation de Chat gpt pour la création du kable}
\CommentTok{\# Extraction des valeurs pertinentes}
\NormalTok{mean\_value }\OtherTok{\textless{}{-}} \FunctionTok{round}\NormalTok{(summary\_price[}\StringTok{"Mean"}\NormalTok{], }\DecValTok{3}\NormalTok{)}
\NormalTok{median\_value }\OtherTok{\textless{}{-}} \FunctionTok{round}\NormalTok{(summary\_price[}\StringTok{"Median"}\NormalTok{], }\DecValTok{3}\NormalTok{)}
\NormalTok{q1\_value }\OtherTok{\textless{}{-}} \FunctionTok{round}\NormalTok{(summary\_price[}\StringTok{"1st Qu."}\NormalTok{], }\DecValTok{3}\NormalTok{)}
\NormalTok{q3\_value }\OtherTok{\textless{}{-}} \FunctionTok{round}\NormalTok{(summary\_price[}\StringTok{"3rd Qu."}\NormalTok{], }\DecValTok{3}\NormalTok{)}
\NormalTok{min\_value }\OtherTok{\textless{}{-}} \FunctionTok{round}\NormalTok{(summary\_price[}\StringTok{"Min."}\NormalTok{], }\DecValTok{3}\NormalTok{)}
\NormalTok{max\_value }\OtherTok{\textless{}{-}} \FunctionTok{round}\NormalTok{(summary\_price[}\StringTok{"Max."}\NormalTok{], }\DecValTok{3}\NormalTok{)}

\CommentTok{\# Création du dataframe}
\NormalTok{result\_df\_ }\OtherTok{\textless{}{-}} \FunctionTok{data.frame}\NormalTok{(}\AttributeTok{Statistique =} \FunctionTok{c}\NormalTok{(}\StringTok{"Moyenne"}\NormalTok{, }\StringTok{"Médiane"}\NormalTok{, }\StringTok{"1er Qu."}\NormalTok{,}
                                         \StringTok{"3ème Qu."}\NormalTok{, }\StringTok{"Min."}\NormalTok{, }\StringTok{"Max."}\NormalTok{, }
                                         \StringTok{"Mode"}\NormalTok{,}\StringTok{"Variance"}\NormalTok{, }
                                         \StringTok{"Ecart{-}type"}\NormalTok{, }
                                         \StringTok{"Coefficient de variation"}\NormalTok{, }
                                         \StringTok{"Skewness"}\NormalTok{, }\StringTok{"Kurtosis"}\NormalTok{),}
  \AttributeTok{Valeur =} \FunctionTok{c}\NormalTok{(mean\_value, median\_value, q1\_value, q3\_value, min\_value,}
\NormalTok{             max\_value, mode\_price, variance\_price, ecart\_type\_price, }
\NormalTok{             co\_price, skewness\_price, kurtosis\_price)}
\NormalTok{)}
\CommentTok{\# Affichage du tableau}
\FunctionTok{kable}\NormalTok{(result\_df\_, }\AttributeTok{caption =} \StringTok{"Résumé des statistiques des prix des jeux"}\NormalTok{)}
\end{Highlighting}
\end{Shaded}

\begin{longtable}[]{@{}lr@{}}
\caption{Résumé des statistiques des prix des jeux}\tabularnewline
\toprule\noalign{}
Statistique & Valeur \\
\midrule\noalign{}
\endfirsthead
\toprule\noalign{}
Statistique & Valeur \\
\midrule\noalign{}
\endhead
\bottomrule\noalign{}
\endlastfoot
Moyenne & 20.587 \\
Médiane & 19.990 \\
1er Qu. & 4.490 \\
3ème Qu. & 29.990 \\
Min. & 0.000 \\
Max. & 69.990 \\
Mode & 0.000 \\
Variance & 307.458 \\
Ecart-type & 17.534 \\
Coefficient de variation & 85.170 \\
Skewness & 0.681 \\
Kurtosis & 2.749 \\
\end{longtable}

\hypertarget{code-de-lhistogramme}{%
\subsection*{Code de l'histogramme}\label{code-de-lhistogramme}}
\addcontentsline{toc}{subsection}{Code de l'histogramme}

\begin{Shaded}
\begin{Highlighting}[]
\FunctionTok{library}\NormalTok{(ggplot2)}

\NormalTok{jeu\_price }\OtherTok{\textless{}{-}}\NormalTok{ jeu}\SpecialCharTok{$}\NormalTok{Price}

\CommentTok{\# Calcul des statistiques descriptives}
\NormalTok{summary\_stats }\OtherTok{\textless{}{-}} \FunctionTok{data.frame}\NormalTok{(}
  \AttributeTok{Statistique =} \FunctionTok{c}\NormalTok{(}\StringTok{"Mean"}\NormalTok{, }\StringTok{"Median"}\NormalTok{, }\StringTok{"1st Qu."}\NormalTok{, }\StringTok{"3rd Qu."}\NormalTok{, }\StringTok{"Min."}\NormalTok{, }\StringTok{"Max."}\NormalTok{, }
                  \StringTok{"Mode"}\NormalTok{, }\StringTok{"Variance"}\NormalTok{, }\StringTok{"Ecart{-}type"}\NormalTok{, }
                  \StringTok{"Coefficient of Variation"}\NormalTok{, }
                  \StringTok{"Skewness"}\NormalTok{, }\StringTok{"Kurtosis"}\NormalTok{),}
  \AttributeTok{Valeur =} \FunctionTok{c}\NormalTok{(}
    \AttributeTok{mean =} \FunctionTok{mean}\NormalTok{(jeu\_price, }\AttributeTok{na.rm =} \ConstantTok{TRUE}\NormalTok{),}
    \AttributeTok{median =} \FunctionTok{median}\NormalTok{(jeu\_price, }\AttributeTok{na.rm =} \ConstantTok{TRUE}\NormalTok{),}
    \AttributeTok{first\_quartile =} \FunctionTok{quantile}\NormalTok{(jeu\_price, }\FloatTok{0.25}\NormalTok{, }\AttributeTok{na.rm =} \ConstantTok{TRUE}\NormalTok{),}
    \AttributeTok{third\_quartile =} \FunctionTok{quantile}\NormalTok{(jeu\_price, }\FloatTok{0.75}\NormalTok{, }\AttributeTok{na.rm =} \ConstantTok{TRUE}\NormalTok{),}
    \AttributeTok{min =} \FunctionTok{min}\NormalTok{(jeu\_price, }\AttributeTok{na.rm =} \ConstantTok{TRUE}\NormalTok{),}
    \AttributeTok{max =} \FunctionTok{max}\NormalTok{(jeu\_price, }\AttributeTok{na.rm =} \ConstantTok{TRUE}\NormalTok{),}
    \AttributeTok{mode =} \FunctionTok{as.numeric}\NormalTok{(}\FunctionTok{names}\NormalTok{(}\FunctionTok{which.max}\NormalTok{(}\FunctionTok{table}\NormalTok{(jeu\_price)))),}
    \AttributeTok{variance =} \FunctionTok{var}\NormalTok{(jeu\_price, }\AttributeTok{na.rm =} \ConstantTok{TRUE}\NormalTok{),}
    \AttributeTok{sd =} \FunctionTok{sd}\NormalTok{(jeu\_price, }\AttributeTok{na.rm =} \ConstantTok{TRUE}\NormalTok{),}
    \AttributeTok{cv =}\NormalTok{ (}\FunctionTok{sd}\NormalTok{(jeu\_price, }\AttributeTok{na.rm =} \ConstantTok{TRUE}\NormalTok{) }\SpecialCharTok{/} 
            \FunctionTok{mean}\NormalTok{(jeu\_price, }\AttributeTok{na.rm =} \ConstantTok{TRUE}\NormalTok{)) }\SpecialCharTok{*} \DecValTok{100}\NormalTok{,}
    \AttributeTok{skewness =}\NormalTok{ moments}\SpecialCharTok{::}\FunctionTok{skewness}\NormalTok{(jeu\_price, }\AttributeTok{na.rm =} \ConstantTok{TRUE}\NormalTok{),}
    \AttributeTok{kurtosis =}\NormalTok{ moments}\SpecialCharTok{::}\FunctionTok{kurtosis}\NormalTok{(jeu\_price, }\AttributeTok{na.rm =} \ConstantTok{TRUE}\NormalTok{) }\SpecialCharTok{{-}} \DecValTok{3}
\NormalTok{  )}
\NormalTok{)}

\CommentTok{\# Créer un histogramme avec ggplot2}
\NormalTok{p }\OtherTok{\textless{}{-}} \FunctionTok{ggplot}\NormalTok{(jeu, }\FunctionTok{aes}\NormalTok{(}\AttributeTok{x =}\NormalTok{ Price)) }\SpecialCharTok{+}
  \FunctionTok{geom\_histogram}\NormalTok{(}\AttributeTok{bins =} \DecValTok{30}\NormalTok{, }\AttributeTok{fill =} \StringTok{"blue"}\NormalTok{, }\AttributeTok{color =} \StringTok{"black"}\NormalTok{) }\SpecialCharTok{+}
  \FunctionTok{geom\_vline}\NormalTok{(}\FunctionTok{aes}\NormalTok{(}\AttributeTok{xintercept =} \FunctionTok{mean}\NormalTok{(Price, }\AttributeTok{na.rm =} \ConstantTok{TRUE}\NormalTok{)), }\AttributeTok{color =} \StringTok{"red"}\NormalTok{, }
             \AttributeTok{linetype =} \StringTok{"dashed"}\NormalTok{, }\AttributeTok{size =} \DecValTok{1}\NormalTok{) }\SpecialCharTok{+}
  \FunctionTok{labs}\NormalTok{(}\AttributeTok{title =} \StringTok{"Histogramme des Prix des Jeux"}\NormalTok{, }\AttributeTok{x =} \StringTok{"Prix"}\NormalTok{, }\AttributeTok{y =} 
         \StringTok{"Nombre de Jeux"}\NormalTok{) }\SpecialCharTok{+}
  \FunctionTok{theme\_minimal}\NormalTok{()}

\CommentTok{\# Afficher l\textquotesingle{}histogramme}
\FunctionTok{print}\NormalTok{(p)}
\end{Highlighting}
\end{Shaded}

\includegraphics{rapport-G17_files/figure-latex/histogramme-1.pdf}

\hypertarget{code-du-nuage-de-point}{%
\subsection*{Code du nuage de point}\label{code-du-nuage-de-point}}
\addcontentsline{toc}{subsection}{Code du nuage de point}

\begin{Shaded}
\begin{Highlighting}[]
\FunctionTok{library}\NormalTok{(ggplot2)}
\FunctionTok{library}\NormalTok{(dplyr)}

\NormalTok{statistique }\OtherTok{\textless{}{-}} \FunctionTok{dbReadTable}\NormalTok{(con, }\StringTok{"statistique"}\NormalTok{)}
\CommentTok{\# Fusion des dataframes \textquotesingle{}statistique\textquotesingle{} et \textquotesingle{}jeu\textquotesingle{} par \textquotesingle{}AppID\textquotesingle{}}
\NormalTok{donnee\_fusion }\OtherTok{\textless{}{-}} \FunctionTok{merge}\NormalTok{(statistique, jeu, }\AttributeTok{by =} \StringTok{"AppID"}\NormalTok{)}

\CommentTok{\# Filtrer les données pour retirer les mois avec peu de joueurs}
\NormalTok{donnee\_fusion }\OtherTok{\textless{}{-}}\NormalTok{ donnee\_fusion }\SpecialCharTok{\%\textgreater{}\%} 
  \FunctionTok{filter}\NormalTok{(AvgPlayers }\SpecialCharTok{\textgreater{}} \DecValTok{10000}\NormalTok{)}

\CommentTok{\# Création d\textquotesingle{}une variable catégorielle pour \textquotesingle{}Metacritic.score\textquotesingle{}}
\NormalTok{donneesMetacriticScore\_groupees }\OtherTok{\textless{}{-}} \FunctionTok{cut}\NormalTok{(}
\NormalTok{  donnee\_fusion}\SpecialCharTok{$}\NormalTok{MetacriticScore, }
  \AttributeTok{breaks =} \FunctionTok{c}\NormalTok{(}\SpecialCharTok{{-}}\ConstantTok{Inf}\NormalTok{, }\DecValTok{50}\NormalTok{, }\DecValTok{75}\NormalTok{, }\DecValTok{90}\NormalTok{, }\ConstantTok{Inf}\NormalTok{),}
  \AttributeTok{labels =} \FunctionTok{c}\NormalTok{(}\StringTok{"\textless{}=50"}\NormalTok{, }\StringTok{"50{-}75"}\NormalTok{, }\StringTok{"75{-}90"}\NormalTok{, }\StringTok{"\textgreater{}90"}\NormalTok{)}
\NormalTok{)}

\CommentTok{\# Ajout de cette nouvelle variable catégorielle au dataframe}
\NormalTok{donnee\_fusion}\SpecialCharTok{$}\NormalTok{ScoreGroup }\OtherTok{\textless{}{-}}\NormalTok{ donneesMetacriticScore\_groupees}

\CommentTok{\# Création du nuage de points avec ggplot2}
\FunctionTok{ggplot}\NormalTok{(donnee\_fusion, }\FunctionTok{aes}\NormalTok{(}\AttributeTok{x =}\NormalTok{ Month, }\AttributeTok{y =}\NormalTok{ AvgPlayers, }\AttributeTok{color =}\NormalTok{ ScoreGroup)) }\SpecialCharTok{+}
  \FunctionTok{geom\_point}\NormalTok{() }\SpecialCharTok{+} 
  \FunctionTok{labs}\NormalTok{(}
    \AttributeTok{x =} \StringTok{"Mois observé d\textquotesingle{}un jeu"}\NormalTok{,}
    \AttributeTok{y =} \StringTok{"Nombre moyen de joueurs (en log(10))"}\NormalTok{,}
    \AttributeTok{title =} \StringTok{"Distribution mensuelle du nombre moyen de joueurs par score Metacritic"}\NormalTok{,}
    \AttributeTok{color =} \StringTok{"Catégories de score Metacritic"}
\NormalTok{  ) }\SpecialCharTok{+} \FunctionTok{coord\_trans}\NormalTok{(}\AttributeTok{y =}\StringTok{"log10"}\NormalTok{)}\SpecialCharTok{+}
  \FunctionTok{theme\_minimal}\NormalTok{() }\SpecialCharTok{+}
  \FunctionTok{theme}\NormalTok{(}
    \AttributeTok{axis.text.x =} \FunctionTok{element\_blank}\NormalTok{(),  }
    \AttributeTok{axis.ticks.x =} \FunctionTok{element\_blank}\NormalTok{(),  }
    \AttributeTok{plot.title =} \FunctionTok{element\_text}\NormalTok{(}\AttributeTok{size =} \DecValTok{12}\NormalTok{)}
\NormalTok{  ) }
\end{Highlighting}
\end{Shaded}

\includegraphics{rapport-G17_files/figure-latex/nuage de point-1.pdf}

\begin{Shaded}
\begin{Highlighting}[]
 \CommentTok{\# coord\_cartesian(ylim = c(1, 2000000))}
\end{Highlighting}
\end{Shaded}

\hypertarget{code-du-diagramme}{%
\subsection*{Code du diagramme}\label{code-du-diagramme}}
\addcontentsline{toc}{subsection}{Code du diagramme}

\begin{Shaded}
\begin{Highlighting}[]
\FunctionTok{library}\NormalTok{(dplyr)}
\FunctionTok{library}\NormalTok{(ggplot2)}
\FunctionTok{library}\NormalTok{(ggpattern)}

\CommentTok{\# Préparation des données}
\NormalTok{jeux\_categorises }\OtherTok{\textless{}{-}}\NormalTok{ jeu }\SpecialCharTok{\%\textgreater{}\%}
  \FunctionTok{inner\_join}\NormalTok{(statistique, }\AttributeTok{by =} \StringTok{"AppID"}\NormalTok{) }\SpecialCharTok{\%\textgreater{}\%}
  \FunctionTok{mutate}\NormalTok{(}
    \AttributeTok{PriceCategory =} \FunctionTok{case\_when}\NormalTok{(}
\NormalTok{      Price }\SpecialCharTok{==} \DecValTok{0} \SpecialCharTok{\textasciitilde{}} \StringTok{"Gratuit"}\NormalTok{,}
\NormalTok{      Price }\SpecialCharTok{\textgreater{}} \DecValTok{0} \SpecialCharTok{\&}\NormalTok{ Price }\SpecialCharTok{\textless{}} \DecValTok{20} \SpecialCharTok{\textasciitilde{}} \StringTok{"Prix \textless{} 20€"}\NormalTok{,}
\NormalTok{      Price }\SpecialCharTok{\textgreater{}=} \DecValTok{20} \SpecialCharTok{\textasciitilde{}} \StringTok{"Prix \textgreater{}= 20€"}
\NormalTok{    ),}
    \AttributeTok{Pattern =} \FunctionTok{ifelse}\NormalTok{(MetacriticScore }\SpecialCharTok{\textless{}} \DecValTok{70}\NormalTok{, }\StringTok{"stripe"}\NormalTok{, }\StringTok{"none"}\NormalTok{)  }
\NormalTok{  ) }\SpecialCharTok{\%\textgreater{}\%}
  \FunctionTok{group\_by}\NormalTok{(PriceCategory, Pattern) }\SpecialCharTok{\%\textgreater{}\%}
  \FunctionTok{summarise}\NormalTok{(}\AttributeTok{moyenne =} \FunctionTok{mean}\NormalTok{(AvgPlayers, }\AttributeTok{na.rm =} \ConstantTok{TRUE}\NormalTok{), }\AttributeTok{.groups =} \StringTok{\textquotesingle{}drop\textquotesingle{}}\NormalTok{)}

\CommentTok{\# Création du graphique avec ggplot2 et ggpattern}
\NormalTok{p }\OtherTok{\textless{}{-}} \FunctionTok{ggplot}\NormalTok{(jeux\_categorises, }\FunctionTok{aes}\NormalTok{(}\AttributeTok{x =}\NormalTok{ PriceCategory, }\AttributeTok{y =}\NormalTok{ moyenne,}
                                  \AttributeTok{pattern =}\NormalTok{ Pattern)) }\SpecialCharTok{+}
  \FunctionTok{geom\_bar\_pattern}\NormalTok{(}\AttributeTok{stat =} \StringTok{"identity"}\NormalTok{, }\AttributeTok{width =} \FloatTok{0.7}\NormalTok{, }\AttributeTok{pattern\_density =} \FloatTok{0.1}\NormalTok{,}
                   \AttributeTok{pattern\_spacing =} \FloatTok{0.02}\NormalTok{) }\SpecialCharTok{+}
  \FunctionTok{scale\_pattern\_manual}\NormalTok{(}
    \AttributeTok{values =} \FunctionTok{unique}\NormalTok{(jeux\_categorises}\SpecialCharTok{$}\NormalTok{Pattern),}
    \AttributeTok{labels =} \FunctionTok{c}\NormalTok{(}\StringTok{"Note \textgreater{}= 70"}\NormalTok{, }\StringTok{"Note \textless{} 70"}\NormalTok{),}
    \AttributeTok{guide =} \FunctionTok{guide\_legend}\NormalTok{(}\AttributeTok{title =} \StringTok{"Catégories de notes Métacritic}\SpecialCharTok{\textbackslash{}n}\StringTok{"}\NormalTok{)}
\NormalTok{  ) }\SpecialCharTok{+}
  \FunctionTok{theme\_minimal}\NormalTok{() }\SpecialCharTok{+}
  \FunctionTok{labs}\NormalTok{(}
    \AttributeTok{title =} \StringTok{"Répartition du nombre moyen de joueurs }
\StringTok{en fonction du prix et de la note Metacritic"}\NormalTok{,}
    \AttributeTok{x =} \StringTok{""}\NormalTok{,}
    \AttributeTok{y =} \StringTok{"Nombre moyen de joueurs"}
\NormalTok{  ) }\SpecialCharTok{+}
  \FunctionTok{theme}\NormalTok{(}
    \AttributeTok{legend.position =} \StringTok{"right"}\NormalTok{,}
    \AttributeTok{legend.title.align =} \FloatTok{0.5}\NormalTok{,}
    \AttributeTok{legend.text =} \FunctionTok{element\_text}\NormalTok{(}\AttributeTok{size =} \DecValTok{9}\NormalTok{),}
    \AttributeTok{legend.title =} \FunctionTok{element\_text}\NormalTok{(}\AttributeTok{size =} \DecValTok{10}\NormalTok{, }\AttributeTok{face =} \StringTok{"bold"}\NormalTok{),}
    \AttributeTok{legend.key =} \FunctionTok{element\_blank}\NormalTok{()  }
\NormalTok{  ) }\SpecialCharTok{+}
  \FunctionTok{coord\_cartesian}\NormalTok{(}\AttributeTok{ylim =} \FunctionTok{c}\NormalTok{(}\DecValTok{0}\NormalTok{, }\DecValTok{60000}\NormalTok{))}

\CommentTok{\# Afficher le graphique}
\FunctionTok{print}\NormalTok{(p)}
\end{Highlighting}
\end{Shaded}

\includegraphics{rapport-G17_files/figure-latex/diagramme-1.pdf}







\end{document}

